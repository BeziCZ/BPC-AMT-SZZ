\section{Chyby měření (rozdělení, výpočet, chyby metody, chyby měřicích přístrojů). Nejistoty měření - typy, výpočet nejistoty A, B, kombinovaná a rozšířená nejistota. Nejistoty nepřímých měření.}
Chyba měřění = přesnost prováděného měření, vyjadřuje se jako rozdíl mezi naměřenou a skutečnou hodnotou měřené veličiny.
\subsection*{Rozdělení}
\begin{itemize}
    \item Chyby měření
          \begin{itemize}
              \item Absolutní
              \item Relativní
          \end{itemize}
    \item Chyba metody
    \item Chyba měřícího přístroje
          \begin{itemize}
              \item Třída přesnosti u analogových měřících přístrojů
              \item Základní chyba číslicových měřících přístrojů
          \end{itemize}
    \item Chyby rušivými vlivy
    \item Podle způsobu výskytu
          \begin{itemize}
              \item systematické
              \item náhodné
              \item hrubé
          \end{itemize}
    \item podle místa vzniku
          \begin{itemize}
              \item Chyby metody měření
              \item Chyba měřícího zařazení
              \item Chyba použitých etalonů
          \end{itemize}
    \item podle příčiny vzniku
          \begin{itemize}
              \item Chyby experimentátora
              \item způsobené rušivými vlivy
              \item vliv přístroje na měřený objekt
          \end{itemize}
\end{itemize}
\newpage
\subsection*{Absolutní chyba měření}
rozdíl mezi naměřenou a kovenčně pravou hodnotou
\begin{equation}
    \Delta_X = X_M - X_P \;\;\;\; [X]
\end{equation}
\subsection*{Relativní chyba měření}
chyba vyjádřená v procentech, častější než absolutní
\begin{equation}
    \delta_X = \frac{\Delta_X}{X_M} \;\;\;\; [\%]
\end{equation}
\subsection*{Korekce}
hodnota, která se musí přičíst k naměřené, aby se získala konvenčně správná hodnota
\begin{equation}
    K_X = X_P - X_M = -\Delta_X \;\;\;\; [X]
\end{equation}
\subsection*{Chyba metody}
vzniká tak, že při výpočtu neuvažujem všechny okolní vlivy nebo zjednodušením/zaokrouhlením výpočtu\\
systematická chyba, která lze eliminovat
\subsection*{Chyba měřícího přístroje}
kalibrací se stanovuje korekční křivka\\
výrobce udává absolutní chybu přístroje\\
\paragraph*{Třída přesnosti pro analogové měřící přístroje}
jedná se o chybu udávanou třídou přesnosti přístroje podle normy ČSN EN 60359(0,05;0,1;0,2;0,3;0,5;1;1,5;2;2,5;3;5)\\
Výpočet třídy přesnosti z maximální absolutní chyby $\Delta_{max}$ na rozsahu $X_R$:
\begin{equation}
    \delta_{TP} = \frac{|\Delta_{max}}{X_R}\cdot 100 \;\;\;\; [\%]
\end{equation}
Jeden přístroj může mět víc tříd nepřesnosti pro různé rozsahy\\
\paragraph*{Vyjadřování chyb číslicového přístroje}
dělí se na:
\begin{itemize}
    \item Základní chybu přístroje
    \item Přídavná chyba přístroje
\end{itemize}
\subparagraph*{Základní chyba}
má dvě složky \\
chyba z měřené hodnoty - nedokonalost nastavení měřících prvků - multiplikativní chyba\\
chyba z rozsahu - posunutí nuly, zbytkové napětí - aditivní chyba\\
Absolutní chyba:
\begin{equation}
    |\Delta_p| = |\Delta_M| + |\Delta_R| = \frac{|\delta_M \cdot X_M|+|\delta_R \cdot X_R|}{100}
\end{equation}
index M - naměřená hodnota \\
index R - rozsah \\
Relativní chyba:
\begin{equation}
    |\delta_p| = |\delta_M| + |\delta_R|
\end{equation}
\\
Způsob zapsání chyby:\\
\textpm(0,05\% z údaje + 0,01\% z rozsahu)\\
\textpm(0,05\% z údaje + 1 digit)
\begin{equation}
    \delta_R = \frac{d}{D}\cdot 100 \;\;\;\;[\%]
\end{equation}
D \dots maximální počet digitů(tj. největší zobrazitelné číslo) \\
d \dots udaný počet digitů posledního místa displeje(většinou 1)\\
\subsection*{Systematické chyby}
přičítá nebo násobí se k měřené hodnotě\\
lze ji určit výpočtem nebo přesnějším měřením\\
lze téměř eliminovat\\

\subsection*{Náhodné chyby}
nelze odstranit korekcí, chová se nepředvídatelně\\
směrodatná odchylka do jisté míry určuje přesnost měření - určuje tvar gaussovy křivky
\begin{equation}
    \sigma = \sqrt{\frac{1}{N}\sum^N_{i=1}(x_i-\overline{x})^2}
\end{equation}
\newpage
\subsection*{Hrubé chyby}
hrubý zásah do procesu měření, výrazně převyšuje rozptyl statistické chyby \\
omyl člověka, výrazná změna podmínek, porucha\\
\begin{figure}[H]
    \includegraphics{images/chyby.png}

\end{figure}

\subsection*{Nejistoty}
Parametr přidružený k výsledku měření, který charakterizuje rozptyl hodnot, které by mohly být důvodně přisuzovány k měřené veličině s určitou pravděpodobností.\\
\begin{itemize}
    \item Standartní nejistota typu A
    \item Standartní nejistota typu B
    \item Kombinovaná standartní nejistota
    \item Rozšířená standartní nejistota
    \item Nejistoty pro nepřímá měření
\end{itemize}

\subsubsection*{Standartní nejistota typu A}
Určuje se  z opakovaných měření.\\
zjišťuje se pomocí výběrového rozptylu hodnot.\\
Mírou nejistoty typu A je výběrový směrodatná odchylka výběrového průměru\\
\begin{equation}
    \overline{x} = \frac{1}{n}\sum^n_{i=1}
\end{equation}
\begin{equation}
    u_a(x) = \sqrt{\frac{1}{n(n-1)}\sum^n_{i=1}(x_i-\overline{x})^2}
\end{equation}
\newpage

\subsubsection*{Standartní nejistota typu B}
Nutnost znát všechny zdroje nejistot, které se dají kvalifikovat, tedy určit jinak, než experimentálně.\\
Zdroje nejistot - kalibrace, stabilita přístrojů, dynamické chyby přístrojů, vnitřní tření\\
Vlivy:
\begin{itemize}
    \item vliv metody - ztráty, interakce s měřeným předmětem, nejistoty použitých konstantn, vlastní ohřev, vlivy reálných parametrů oproti ideálním
    \item vliv operátora - nedodržení metodik, elektrostatické pole, tepelné vyzařování
    \item ostatní vlivy - náhodné omyly při odečtu nbo zápisu hodnot, globální vlivy
\end{itemize}
Dají se vypsat z datasheetů nebo odhadovat.\\
Výpočet nejistoty B:
\begin{equation}
    u_{BZ}(x) = \frac{\Delta_{zmax}}{\kappa}
\end{equation}
Celková nejistota typu B je geometrický součet nejistot jednotlivých zdrojů
\begin{equation}
    u_B(x) = \sqrt{\sum^n_{z=1}u^2_{Bz}(x)}
\end{equation}
Koeficient tvaru rozložení $\kappa$ se určuje z tabulky:
\begin{figure}[H]
    \includegraphics[scale = 1]{images/koef_rozlozeni.png}
\end{figure}
\newpage
Vzorový výpočet nejistoty typu B pro analogový přístroj:
\begin{figure}[H]
    \centering
    \includegraphics[scale = 1]{images/nejistotaB_analog.png}
\end{figure}
Vzorový výpočet pro nejistotu typu B pro digitální přístroj:
\begin{figure}[H]
    \centering
    \includegraphics[scale = 1]{images/nejistotaB_digital.png}
\end{figure}

\subsubsection*{Kombinovaná standartní nejistota}
Geometrický součet nejistoty typu A a B
\begin{equation}
    u_C(x) = \sqrt{u_A^2(x)+u_B^2(x)}
\end{equation}
\newpage

\subsubsection*{Rozšířená standartní nejistota}
Kombinovaná standartní nejistota odpovídá intervalu, ve kterém by se výsledky pohybovaly s pravděpodobností 68\%.\\
Rozšířená standartní nejistota pomocí koeficientu tuto pravděpodobnost zvyšuje, koeficient se volí podle tabulky:
\begin{figure}[H]
    \centering
    \includegraphics[scale = 1]{images/nejistota_roz.png}
\end{figure}
Nejistota vzniká Kombinovanou standartní nejistotou vynásobenou tímto koeficientem.
\begin{equation}
    U(x) = k_r\cdot u_C(x)
\end{equation}

\subsection*{Nepřímá měření}
Výsledek měření je určen výpočtem na základě známých fyzikálních zákonů z přímých měření vstupních veličin.\\
Výstupní veličina je funkcí průměrných vstupních veličin.\\
\subsubsection*{Nejistota nepřímého měření nekorelovaných vstupních veličin}
Odhad veliiny Y, která je funkcí nekorelovaných vstupních veličin X je dána vztahem:
\begin{equation}
    u(\overline{y}) = \sqrt{\sum^N_{i=1}\left(\frac{\partial f}{\partial X_i}u(\overline{x_i})\right)^2} = \sqrt{\sum^N_{i=1}A_i^2\cdot u^2(\overline{x_i})}
\end{equation}
\begin{equation}
    A_i = \frac{\partial f(X_1,X_2,\dots ,X_N)}{\partial X_i}
\end{equation}
$f$ \dots funkční závislost pro výstupní veličinu $Y$ \\
$A_i$ \dots citlivostní koeficient\\
$u(\overline{x_i})$ \dots nejistota odhadu vstupní veličiny $X_i$\\
\newpage
\subsubsection*{Nejistota nepřímého měření vzájemně korelovaných vstupních veličin}
Odhad veličiny $Y$, která je funkcí vzájemně korelovaných vstupníc veličin $X$, zahrnuje rovněž kovariance mezi jednotlivými odhady vstupních veličin a ty rovněž přispívají k výsledné nejistotě podle vztahu:
\begin{equation}
    u(\overline{y}) = \sqrt{\sum^N_{i=1}A_i^2\cdot u^2(\overline{x_i})+2\sum^N_{i=2}\sum^{N-1}_{k<i}A_i\cdot A_k \cdot u^2(\overline{x_i},\overline{x_k})}
\end{equation}
$u(\overline{x_i},\overline{x_k})$ \dots kovariace mezi navzájem korelovanými odhady vstupních veličin $\overline{x_i},\overline{x_k}$


\section{Analogově-číslicové převoddníky pro měřicí techniku - rozdělení, princip základních typů AD převodníků, vlastnosti, použití}
Převod analogové diskrétní proměnné veličiny(napětí) na diskrétní údaj vyjádřený číslem.\\
Diskretizace analogového signálu:
\begin{itemize}
    \item V amplitudě - kvantování
    \item V čase - vzorkování(vzorkovací teorém - vzorkovací frekvence musí být větší, než 2* maximální požadovaná přenášená frekvence)
    \item Kódování
\end{itemize}
Kvantizační chyba - chyba zaokrouhlování na nejmenší jednotku přístroje. Při časové změně měřené veličiny vzniká kvantovací šum\\
\begin{figure}[H]
    \centering
    \includegraphics*[scale = 1]{images/adc_prevodni_char.png}
    \caption*{Převodní charakteristika měřicího přístroje}
\end{figure}
$K_i$ ideální převodní charakteristika\\
$K_S$ je skutečná převodní charakteristika\\
Chyby, které se vyskytují:
\begin{itemize}
    \item chyba nuly - nenulový výstup pro nulové napětí
    \item chyba konstanty - jiný sklon charakteritiky
    \item chyba linearity - průměrná charakteristika není lineární
\end{itemize}
\newpage
Kritéria hodnocení AD převodníku:
\begin{itemize}
    \item rozlišovací schopnost
    \item krok kvantování
    \item chyba kvantování
    \item rychlost převodu
    \item přesnost
    \item stabilita
\end{itemize}
\subsection*{Rozdělení AD převodníků}
\begin{itemize}
    \item Komparační
          \begin{itemize}
              \item Metoda paralelního porovnání
              \item Metoda sérioparalelního porovnání
          \end{itemize}
    \item Kompenzační
          \begin{itemize}
              \item Kompenzační metoda
              \item Metoda postupné aproximace
          \end{itemize}
    \item Integrační
          \begin{itemize}
              \item Integrační metoda (metoda dvojté integrace)
              \item Převod napětí na kmitočet metodou jednoduché integrace
          \end{itemize}
    \item Sigma-delta modulace
\end{itemize}

\subsection*{Komparační A/D převodník(FLASH)}
Srovnává měřené napětí s různě nastavenýma hodnotama referenčního napětí najednou.\\
Vzorkovací frekvence desítky MHz až 3 GHz\\
krátká doba převodu(ns) - nejrychlejší\\
drahý, není odolný vůči sériovému rušení
\begin{figure}[H]
    \includegraphics*[scale = 1.3]{images/adc_komparacni.png}
\end{figure}
\newpage
\subsection*{Kompenzační A/D převodník}
měřené konstantní napětí se porovnává se známým stavitelným kompenzačním napětím - nastavuje se tak, aby se blížilo měřenému\\
nejde nastavit naprostá shoda, maximálně zanedbatelný rozdíl - nastavuje se postupně\\
Typy:
\begin{itemize}
    \item Přírůstkový převodník
    \item Sledovací převodník - umožňuje čítat i dolů
    \item Aproximační převodník - nahrazen registrem
\end{itemize}
\subsubsection*{Metoda postupné aproximace}
Hodnota napětí vyjádřena přirozeným binárním číslem\\
při převodu se určuje bit po bitu od největšího po nejmenší\\
8-16 bitové\\
vzorkovací frekvence 10kHz - 3MHz\\
\begin{figure}[H]
    \includegraphics*[scale = 1]{images/adc_kompenzacni_aprox.png}
\end{figure}
\newpage

\subsection*{Integrační A/D převodník}
vhodné a často používané pro číslicové voltmetry\\
výsledkem integrace je střední hodnota\\
Parametry:
\begin{itemize}
    \item 10-27 bitové
    \item vzorkovací frekvence 0,1Hz až 1kHz
    \item pomalé
    \item velká přesnost, eliminace sériového rušení
\end{itemize}
Princip:
\begin{itemize}
    \item dvojí(dvoutaktní) integrace
    \item neznámé vstupní napětí porovnáváno s referenčním opačné polarity
    \item po skončení druhého taktu stav čítače hodinových pulsů N udává velikost neznámého napětí
\end{itemize}

\begin{equation}
    N = f_hT_2 = f_hT_1\cdot \frac{U_{vst}}{Z_{REF}} = N_c \cdot \frac{U_{vst}}{Z_{REF}}
\end{equation}
\begin{figure}[H]
    \includegraphics*[scale = 1]{images/adc_integrac.png}
\end{figure}

\subsubsection*{Převodník napětí na kmitočet}
Převod měřeného napětí na kmitočet a ten pak změřit číslicově.\\
převáděné napětíje trvale připojeno na vstup integrátoru, vyvolává změnu napětí na výstupu integrátoru, které se v komparátoru porovnává s konstantním napětím\\
při překročení hodnoty se změní na opačné výstupní napětí integrátoru\\
kmitočet výstupních impulsů je pak úměrný měřenému napětí
\begin{figure}[H]
    \includegraphics*[scale = 1]{images/adc_V_na_f.png}
\end{figure}

\subsection*{Sigma-delta modulace}
Dle taktu Uc vzorkování na komparátor\\
Uo obdelníkové, dle trvání vyšší a nižší hodnoty vyjadřuje Ux\\
dobré rozlišení o na 16 bit při 100kHz, dlouhá odezva
\begin{figure}[H]
    \includegraphics*[scale = 1]{images/adc_sigma_delta_schema.png}
\end{figure}

\begin{figure}[H]
    \includegraphics*[scale = 1]{images/adc_sigma_delta_graf.png}
\end{figure}


\subsection*{Vlastnosti}
\begin{itemize}
    \item Rozlišovací schopnost - dána počtem rozlišitelných úrovní analogového signálu. Pro n-bitový převodník je to $2^n$ úrovní
    \item Krok kvantování(citlivost) - rozdíl dvou hodnot vstupního analogového napětí, kdy nastává přechod od jednoho číslicového výstupu k druhému
    \item Chyba kvantování - maximální rozdíl mezi hodnotou analogové veličiny a hodnotou odpovídající danému ódovanému slovu(obvykle polovina kroku kvantování)
    \item Rychlost převodu
    \item Přesnost - dána chybou převodníku
    \item Stabilita - vyjadřuje stálost převodníku při působení různých rušivých vlivů
\end{itemize}

\subsection*{Použití}
Zpracování analogových signálů v digitálních přístrojích.

\section{Měření napětí a proudu. Změna rozsahů voltmetru a ampérmetrů. Rušení u měřicích přístrojů.}

\subsection*{Měření napětí}
\subsubsection*{voltmetry}
zapojení paralelně k zátěži\\
ideální voltmetr - nekonečný vnitřní odpor, reálně digitální 10MOhm, analogový 5kOhm/V \\
\subsubsection*{Přístroje pro měření napětí}
\begin{itemize}
    \item analogové
          \begin{itemize}
              \item reagují na hodnotu měřeného napětí výchylkou ručky
              \item obsahují převodník pro změny rozsahů
          \end{itemize}
    \item analogové elektronické
    \item číslicové(stejnosměrný voltrmetr je základem multimetru)
\end{itemize}
\subsubsection*{Měření stejnosměrného napětí(DC)}
na DC napětí se převádí pomocí měřicích převodníků všechny ostatní měřené veličiny\\
v praxi bývá problém s napěťovým offsetem a jeho driftem v jednosměrných předzesilovačích ve voltmetrech\\
u měření vysokých napětí(v řádu kV) problém s bezpečností a izolačními možnostmi materiálu\\
\subsubsection*{Měření střídavého napětí(AC)}
požadovaný parametr obvykle efektivní hodnota\\
Dělí se podle šířky pásma:
\begin{itemize}
    \item Úzkopásmové - selektivní, místo voltmetrl obvykle spektrální analyzátor, oblast vysoké frekvence
    \item Širokopásmové - nejpoužívanější, oblast nízké frekvence (v řádu Hz až kHz)
    \item Multimetry - napěťové rozsahy od mV po stovky V
\end{itemize}
\newpage

\subsubsection*{Měřicí transformátory napětí}
k převodu měřené veličiny na velikost vhodnou k měření a ke galvanickému oddělení měřicích obvodů a přístrojů od vysokého napětí
Transformační poměr pro idealizovaný stav:
\begin{equation}
    p_U = \frac{U_1}{U_2} = \frac{N_1}{N_2}
\end{equation}
\begin{figure}[H]
    \centering
    \includegraphics*[scale = 2]{images/napeti_transformator.png}
\end{figure}
Princip:
\begin{itemize}
    \item primární vinutí M-N se připojuje (MTN-měřicí transformátor proudu)do obvodu paralelně
    \item sekundární vinutí m-n obsahuje měřicí obvod představující vysokou impedanci
    \item MTN musí pracovat v blízkosti stavu narpázdno
    \item se chová jako zdroj s malým vnitřním odporem
\end{itemize}
Vlastnosti:
\begin{itemize}
    \item oproti předřadným odporům velmi malá spotřeba
    \item hodnota sekundárního napětí obvykle 100V
    \item způsobují chybu převodu - udává se třídou přesnosti
    \item způsobují chybu fáze - nutno počítat při měření výkonu
    \item nelze použít, kde by docházelo k magnetování jádra stejnosměrnou složkou průběhu - bo cívky
\end{itemize}
\subsubsection*{Typy rušení}
Na vstupní svorky stejnosměrného voltmetru se mohou k měřenému napětí přičítat rušívá napětí.\\
Rušivá napětí:
\begin{itemize}
    \item Stejnosměrná
    \item Střídavá
    \item Náhodná
\end{itemize}
Nejčastěji rušení střídavé od napájecí sítě s frekvencí 50 Hz.\\
\newpage
Sériové rušení:
\begin{itemize}
    \item Zdroj rušivého signálu je zapojený sériově s měřenou veličinou
    \item Sčítá se s měřeným
\end{itemize}
Odstrańnuje se buď účinným filtrem dolní propusti před číslicovým voltmetrem, nebo některé A/D převodníky dokáží sériové rušení potlačit automaticky.\\
Souhlasné rušení:
\begin{itemize}
    \item při měření napětí mezi dvěma body elektrického obvodu mohou nastat situace, kdy tyto body mají proti společné svorce určité napětí
    \item je vyvoláno rozdílem potenciálu země voltmetru a země měřeného objektu
    \item potlačuje se konstrukcí přístroje - rozpojením zemní smyčky použitím multimetru s bateriovým nanápajením
\end{itemize}

\subsubsection*{Číslicové voltmetry}
Základní měřicí přístroj v praxi, víc se ale využívají multimetry\\
Základní části číslicového voltmetru:
\begin{figure}[H]
    \includegraphics*[scale = 1.5]{images/cislicovy_voltmetr.png}
\end{figure}
Vstupní část umožňuje volbu dílčích měřicích rozsahů. Převádí velkost měřeného napětí na velikost vyhovujícího převodníku napětí na číslo.\\
Dočasné uchování naměřených dat.\\
Parametry:
\begin{itemize}
    \item počet míst zobrazovače (3 až 8\textonehalf)
    \item počet měřicích rozsahů (4 až 6, zkl. rozsah 1 nebo 10V)
    \item přesnost
    \item rozlišovací schopnost - velikost napětí na vstupu voltmwtru, která způsobí změnu údaje voltmetru o 1 na posledním místě číslicového zobrazovače při nejnižším rozsahu
    \item maximální měřené napětí - obvykle 1000 V DC a 750 V AC
    \item rychlost měření
    \item časová stálost
    \item odolnost proti rušení
    \item vstupní impedance
    \item typ použitého AD převodníku
    \item kmitočtový rozsah(50kHz až 2 MHz)
\end{itemize}

Chyby číslicových měřicích přístrojů:
\begin{itemize}
    \item pevné chyby - nezávislé na velikosti vstupního signálu(vyjařují se v \% z měřicícho rozsahu), chyby způsobené:
          \begin{itemize}
              \item Driftem napěťové nesymetrie vstupního zesilovače
              \item Vnitřním šumem přístroje
              \item Zbytkovým napětím spínačů
              \item Chyba kvantováním
          \end{itemize}
    \item Chyby úměrné měřené hodnotě(projevují se multiplikativně, taky v \% z rozsahu) - způsobené chybami:
          \begin{itemize}
              \item přenosu zesilovačů
              \item vstupní děliče
              \item napětí referenčního přístroje
          \end{itemize}
\end{itemize}
Změna rozsahu voltmetru:\\
Rozsah se zvětší zapojením předřadného rezistoru do série - předřadník:\\
\begin{figure}[H]
    \includegraphics*[scale = 1.5]{images/predradnik.png}
\end{figure}

\subsection*{Měření proudu}

Pomocí ampérmetrů - zapojení do série s měřeným obvodem.\\
Ideální ampérmetr má nulový opor, když nějaký má, je to chyba metody.\\
Velikost měřených proudů:
\begin{figure}[H]
    \includegraphics*[scale = 1.2]{images/Imereni.png}
\end{figure}
Měření v rozsahu $\mu A$ až jednotek A.\\
U nižších proudů problém se vstupními proudy a jejich driftem v předzesilovači.\\
U vyšších proudů problém se snímáním proudu, lze použít speciální proudové sondy a nepřímé metody.\\
\newpage

\subsubsection*{Měření malých proudů}
Je potřeba použít měřící zesilovače nebo použít převodník proudu na napětí.\\
To umožňuje měření malých proudů bez úbytku napětí, vstupní odpor se bíží 0 Ohm, bývají součástí číslicových ampérmetrů.\\
Velmi malé proudy se převádí pomocí velkého odporu napětí, které se měří mikrovltmetrem s modulačním zesilovačem.\\
\subsubsection*{Měření velkých proudů}
Proudový transformátor:
\begin{itemize}
    \item pouze pro střídavý proud
    \item úzký frekvenční rozsah
    \item levné, jednoduché
\end{itemize}
Hallova sonda - polovodičový snímač magnetického proudu
\begin{figure}[H]
    \includegraphics*[scale = 1]{images/hallova_sonda.png}
\end{figure}

Vlastnosti:
\begin{itemize}
    \item použití pro DC aj AC proud
    \item rozsah měřených proudů od mA po stovky kA
    \item do frekvence 25 kHz
    \item přesnost až 1\%
    \item digitální výstup
    \item vyšší cena
\end{itemize}
\newpage
Měřicí transformátor proudu\\
princip:
\begin{itemize}
    \item primární obvod měřicícho transformátoru proudu(MTP) je zapojen do série se zdrojem
    \item sekundární vinutí se uzavirá přes měřicí přístroj
\end{itemize}
\begin{figure}[H]
    \includegraphics*[scale = 1.3]{images/proud_transformator.png}
\end{figure}

Měřicí transformátor proudu musí vždy pracovat v blízkosti stavu nakrátko.
\begin{itemize}
    \item před odpojením měřicích přístrojů je nutné sekundární vinutí zkratovat
    \item protéká-li primárním vinutím proud, nesmí být sekundární vinutí rozpojeno
\end{itemize}

Parametry:
\begin{itemize}
    \item transformační poměr
    \item jmenovitá hodnota sekundárního proudu
    \item povolená sekundární zátěž
\end{itemize}
Charakteristické údaje:
\begin{itemize}
    \item Dynamický proud MTP - největší amplituda zkratového proudu při sekundárním vinutí nakrátko - nesmí dojít k mechanickému nebo elektrickému poškození
    \item Tepelný proud - efektivní hodnota primárního proudu po dobu 1 sekundy při sekundárním vinutí nakrátko - nesmí dojít k poškození, závisí na průřezu primárního vinutí
    \item Nadproudové číslo - násobek jmenovitého proudu, kdy chyba převodu dosáhne 10\% při jmenovité zátěži a účiníku
\end{itemize}
Přesnost převodu závisí na:
\begin{itemize}
    \item velikosti měřeného proudu
    \item kmitočtu měřeného proudu
    \item tvaru a vlastnostech jádra a vinutí
    \item velikost a charakteru sekundární impedance
\end{itemize}
Parazitní vlivy:
\begin{itemize}
    \item rozptylová indukčnost obou vinutí
    \item ztráty v železe vlivem vířivých proudů
\end{itemize}
\newpage
Výhody:
\begin{itemize}
    \item lze transformovat nejen z větší na menší
    \item spotřeba měřicích obvodů se s měřicími transformátory nemění se změnou rozsahu
\end{itemize}
Nevýhoda - nelze transformovat střídavé průběhy se stejnosměrnou složkou

Měření proudu pomocí multimetru:
\begin{itemize}
    \item převod proudu na napětí - požadavek co nejmenší impedance
    \item typy převodníků:
          \begin{itemize}
              \item pasivní převodník - přesný snímací rezistor, přepočet měřeného proudu podle poměrů vstupního odporu ampérmetru a bočníku
              \item aktivní převodník - využití OZ
          \end{itemize}
\end{itemize}

\subsection*{Rušení u měřicích přístrojů}
\subsubsection*{Mechanické vlivy}
\begin{itemize}
    \item tření - zvětšením direktivního momentu spirálových pružin, větší spotřeba, menší citlivost přístroje
    \item Otřesy - dát přístroj na měkkou podložku
    \item poloha přístroje - měly by se používat jen v poloze uvedené v seznamu
\end{itemize}

\subsubsection*{Teplota}
\begin{itemize}
    \item Mění se odpor měřicích cívek, předřadníků či bočníků a magnetická indukce permanentních magnetů
    \item předřadníky a bočníky dáváme do přední části přístroje, aby neovlivňovaly měčicí cívky, použijeme materiál, který má nepatrnou závislost na teplotě, chladící otvory
\end{itemize}
\subsubsection*{Vnější elektromagnetické pole}
\begin{itemize}
    \item působí zejména na přístroje, které pracují s vlastním slabým polem
    \item bráníme stíněním měřicího ústrojí krytem z feromagnetického materiálu
\end{itemize}

\subsubsection*{Kmitočet}
\begin{itemize}
    \item změna kmitočtu způsobí změnu údaje u těch přístrojů, jejichž pohybový moment na kmitočtu závisí
    \item u každého přístroje se řeší individuálně
\end{itemize}


\subsection{Rušení}
Další možné rozdělení rušení je 
\begin{itemize}
    \item Stejnosměrná
    \item střídavá
    \item Náhodná
    \item v sérii
    \item proti zemi
\end{itemize}

\subsubsection{v sérii}
\begin{itemize}
    \item dá se potlačit filtry, (problém je v tom že nejvic rušní je na 50 Hz, tzn časová kostatna filtru je 1s, takže jeho použití pomalí měření)
    \item další možnost je použít dobrý AD převodník
    \begin{itemize}
        \item ANO itegarční převodník s mezipřevodem na frekvenci
        \item ANO integrační převodní  s dvojnásoubnou integarcí
        \item ne integrační převodník s metipřevodem na časový interval
        \item osattní spíš taky ne
    \end{itemize}
    \item určuje se činnitel potlačení sériového rušení
        \begin{equation}
            H=SMRR= 20log\frac{U_{SM}}{\Delta U_{ch}}
        \end{equation}
        Usm rušivé napětí napětí $ \Delta U_{ch} $ změna údaje voltmetru
\end{itemize}

\subsubsection{Proti zemi}
\begin{itemize}
    \item odsatranění oddělením měřící země od napájecí
    \item extra svorka na zemění
    \item polovoucí vstup
\end{itemize}

\includegraphics{images/ruseni.png}
\newpage

\section{Měření výkonu v jednofázové a třífázové soustavě. Wattmetry - princip, typy vlastnosti.}
Výkon stejnosměrného proudu:
\begin{equation}
    P = U\cdot I \;\;\;\; [W]
\end{equation}
Výkon střídavého proudu:
\begin{itemize}
    \item Okamžitý výkon(časový okamžik) $p(t) = u(t)\cdot i(t)$   [W]
    \item střední hodnota výkonu(časový interval) $P = \frac{1}{T}\int_0^Tp(t)dt = \frac{1}{T}\int_0^Tu(t)\cdot i(t)dt$    [W]
\end{itemize}

Výkon harmonického proudu a napětí s fázovým posuvem:
\begin{itemize}
    \item Činný výkon    $P = U\cdot I \cdot cos \upvarphi $    [W]
    \item Jalový výkon   $Q = U\cdot I \cdot sin \upvarphi $    [VAr]
    \item Zdánlivý výkon  $S = U \cdot I$                     [VA]
\end{itemize}
Výkon neharmonických proudů a napětí: $P = U_0\cdot I_0 + \sum^{k=n}_{k=1}U_k\cdot I_k \cdot cos \upvarphi_k$     [W]
\subsection*{Měření výkonu v jednofázové soustavě}
Metody:
\begin{itemize}
    \item elektromechanické wattmetry s elektrodynamickým ústrojím
    \item elektronické wattmetry - nejde odečíst proud/napětí na zátěži, přidává se voltmetr/ampermetr
    \item měření třemi voltmetry
    \item měření třemi ampérmetry
\end{itemize}

\subsubsection*{Metoda wattmetrem}
dva způsoby, oba zatíženy chybou metody
\begin{figure}[H]
    \includegraphics*[scale = 1]{images/wattmetry2.png}
\end{figure}
\newpage

\subsubsection*{Metoda třemi voltmetry}
Nepřímá metoda pro měření malých činných výkonů na malých impedancích zátěže Z
\begin{figure}[H]
    \includegraphics*[scale = 1.5]{images/voltrmetry3.png}
\end{figure}

\subsubsection*{Metoda třemi ampérmetry}
\begin{figure}[H]
    \includegraphics*[scale = 1.5]{images/ampermetry3.png}
\end{figure}

\subsubsection*{Měření jalového výkonu}
\begin{figure}[H]
    \includegraphics*[scale = 1.3]{images/jalovec.png}
\end{figure}
Charakter zátěže spotřebiče:
\begin{itemize}
    \item Induktivní - kladný jalový výkon
    \item Kapacitní - záporný jalový výkon
\end{itemize}

\subsubsection*{Měření zdánlivého výkonu}
Jedině nepřímo, počítá se z proudu a napětí


\subsection*{Měření výkonu v třífázové soustavě}
Všechna zapojení vyžadují znalost sledu fází:
\begin{figure}[H]
    \includegraphics*[scale = 1.2]{images/fazovyDiagram.png}
\end{figure}
Sled fází je určem pořadím, v němž veličiny trojfázové soustavy nabývají svého maxima.\\

Fázové napětí - napětí mezi fázovým a nulový vodičem\\
Sdružené napětí - napětí mezi dvěma fázovými vodiči
\begin{figure}[H]
    \includegraphics*[scale = 1.2]{images/sdruzeneNapeti.png}
\end{figure}

\subsubsection*{Základní pojmy}
\begin{itemize}
    \item Souměrná zátěž - Síť je zatížena souměrným spotřebičem, protéká v každé fázi stejný proud(motor)
    \item Nesouměrná zátěž - spotřebič odebírá v každé fázi jiný proud(oblouková pec)
    \item Obecná (nesymetrická) soustava - má obecné velikosti fázových napětí a obecné úhly mezi nimi - čtyřvodičová má přístupný nulový vodič, třívodičová ne
    \item Symetrická soustava - zvláštní případ obecné soustavy, fázová napětí mají stejnou velikost i úhel, sdružené napětí je $\sqrt{3}$ krát větší, než fázové
\end{itemize}

\subsubsection*{Matematický popis}
Souměrná soustava:
\begin{itemize}
    \item efektivní hodnota: $U_1 = U_2 = U_3 = U_f$
    \item fázový posuv: $\upvarphi = 2\pi /3$
\end{itemize}
$u_1 = \sqrt{2}U_fsin(\omega t +\upvarphi)$\\
$u_2 = \sqrt{2}U_fsin(\omega t -120\degree +\upvarphi)$\\
$u_3 = \sqrt{2}U_fsin(\omega t -120\degree +\upvarphi)$\\


Volba  metody záleží na tom, jestli je měřená zátěž ne/symetrická a jestli je trojfázová síť tří nebo čtyřvodičová\\

\subsubsection*{Blondelův teorém}
V n-vodičové soustavě se činný výkon zátěže správně měří nejméně n-1 wattmetry

\subsubsection*{Symetrická zátěž, čtyřvodičová symetrická soustava}
\begin{figure}[H]
    \includegraphics*[scale = 1.2]{images/sym_4_sym.png}
\end{figure}


\subsubsection*{Symetrická zátěž, třívodičová symetrická soustava}
\begin{figure}[H]
    \includegraphics*[scale = 1.2]{images/sym_3_sym.png}
\end{figure}

\subsubsection*{Neymetrická zátěž, čtyřvodičová soustava}
\begin{figure}[H]
    \includegraphics*[scale = 1.2]{images/nesym_4.png}
\end{figure}
\newpage

\subsubsection*{Nesymetrická zátěž, třívodičová soustava}
\begin{figure}[H]
    \includegraphics*[scale = 1.2]{images/nesym_3.png}
\end{figure}

\subsubsection*{Aronovo zapojení}
Slouží k měření rinného výkonu v třívodičové soustavě bez středového vodiče.\\
Lze použít pro souměrnou i nesouměrnou zátěž.\\
Napěťové obvody wattmetrl jsou připojeny na sdružená napětí.
\begin{figure}[H]
    \includegraphics*[scale = 1.2]{images/Aron.png}
\end{figure}

\subsubsection*{Měření Jalového výkonu}
Nutnost zajistit zpoždění fázoru napětí na napěťové cívce každého wattmetru o 90° oproti fázoru napětí při měření činného výkonu - připojit sdružená napětí na napěťové cívky\\
\begin{figure}[H]
    \includegraphics*[scale = 1]{images/jalonen.png}
\end{figure}
\newpage

\subsection*{Wattmetry}
\begin{figure}[H]
    \includegraphics*[scale = 1]{images/wattSchemata.png}
\end{figure}
Rozdělení:
\begin{itemize}
    \item Z funkčního hlediska:
          \begin{itemize}
              \item Průchozí - zapojuje se mezi zdroj a zátěž, měří výkon předávaný zdrojem do zátěže
              \item Pohlcovací - měří výkon předávaný zdrojem do odporové zátěže obsažené ve wattmetru připojeného ke zdroji
          \end{itemize}
          \begin{figure}[H]
              \includegraphics*[scale  = 1]{images/wattTypy.png}
          \end{figure}
    \item Podle způsobu měření a charakteru údaje:
          \begin{itemize}
              \item Analogové
              \item Digitální
          \end{itemize}
\end{itemize}

\subsubsection*{Průchozí wattmetr}
Použití:
\begin{itemize}
    \item nízkofrekvenční oblast
    \item měření harmonických průběhů - energetika
    \item měření neharmonických průběhů - měření výkonu i harmonicky zkresleného průběhu v síti
\end{itemize}


\begin{figure}[H]
    \includegraphics*[scale  = 1]{images/wattPruchozi.png}
\end{figure}

Rozdělení:
\begin{itemize}
    \item Analogové wattmetry
    \item Číslicové wattmetry s analogovým zpracováním signálů
    \item Číslicové wattmetry s číslicovým zpracováním signálů
\end{itemize}

Analogový wattmetr:
\begin{figure}[H]
    \includegraphics*[scale  = 1.3]{images/wattAnalog.png}
\end{figure}

Číslicový wattmetr s analogovým zpracováním signálu
\begin{itemize}
    \item Smodulační násobičkou
          \begin{itemize}
              \item chyba v desetinách procenta
              \item frekvenční pásmo 0-10kHz
              \item Princip:
                    \begin{figure}[H]
                        \includegraphics*[scale  = 1]{images/wattAnalogPrincip.png}
                    \end{figure}
          \end{itemize}
          \newpage

    \item Hallova násobička
          \begin{itemize}
              \item Působí-li na plochu destičky z polovodičového materiálu magnetické pole o indukci B a protéká-li destičkou proud i, objeví se na zbývajících hranách hallovo napětí $U_H$
              \item Napětí $U_H$ je nutno integrovat, abychom získali stejnosměrnou hodnotu úměrnou činnému výkonu P
          \end{itemize}
\end{itemize}
\begin{figure}[H]
    \includegraphics*[scale  = 1.3]{images/wattHallovaNasobicka.png}
\end{figure}

Číslicový wattmetr s číslicovým zpracováním signálu
\begin{itemize}
    \item přechod od předchozího typu doplněný o AD převodník a nahrazení analogové násobičky procesorem
\end{itemize}
\begin{figure}[H]
    \includegraphics*[scale  = 1.3]{images/wattCislicovySchema.png}
\end{figure}

\subsubsection*{Pohlcovací wattmetry}
Určeny k měření výkonu dodávaného zdrojem do odporové zátěže představované samotným wattmetrem.\\
Výkon se určuje buď z napětí na zátěži, nebo z tepelného účiníku.\\
Rozsah MHz až destíky GHz.\\
Většinou ve formě externích modulů k PC nebo jako doplněk spektrálních analyzátorů.\\
Analogový nebo číslicový údaj.\\
Odporová zátěž:
\begin{itemize}
    \item Měřicí rozsah $1 \mu W$ až $1 kW$
    \item pro vyšší frekvence se používá zatěžovací rezistor zakončující koaxiální vedení
\end{itemize}
\begin{figure}[H]
    \includegraphics*[scale  = 1.3]{images/wattOdporovaZatez.png}
\end{figure}

Voní zátěž:
\begin{itemize}
    \item pro střední a větší výkony(desítky W až kW) - přivedený měřený výkon protékající vodu ohřívá, měří se průtok a oteplení vody
    \item buď v koaxiálním provedení(do 3 GHz) nebo vlnovodném provedení(nad 2GHz)
\end{itemize}
\begin{figure}[H]
    \includegraphics*[scale  = 1]{images/wattVodniZatez.png}
\end{figure}

Zátěž v podobě vedení:
\begin{itemize}
    \item Pro měření menších výkonů(desítky W)
    \item použití "suché" zátěže ve formě bezodrazného zakončení koaxiálního vedení nebo vlnovodného vedení
    \item telpota zátěže, která závisí na rozptylovém výkonu, se měří pomocí:
          \begin{itemize}
              \item termočlánkového snímače
              \item diodového snímače
              \item termistorového snímače
          \end{itemize}
\end{itemize}


\section{Číslicové osciloskopy - princip, vlastnosti, jejich příslušenství}
Základní měřicí přístroj umožňující zobrazi časově proměnný signál obvykle ve formě grafu
umožňují současná záznam několika kanálů před příchodem spouštěcího pulsu, snané zachycení jednorázových dějů, výstup na tiskárnu, uložení průběhu do paměti a jejich číslicové zpracování, přenos dat do PC.\\
\subsection*{Princip}
\begin{itemize}
    \item Pracují na principu vzorkování sledovaného jevu
    \item u číslicového osciloskopu není paprsek bezprostředně vychylován vstupním signálem - plynulý vstupní signál je rozložen na diskrétní měřicí body - vzorky, které jsou digitalizovány, uloženy do paměti a znovu skládány na displej 
    \item víc tlačítek, než analog(ukládání paměti, meření zobrazovaného signálu)
\end{itemize}
\begin{figure}[H]
    \includegraphics*[scale = 1]{images/osciloskopSchema.png}
\end{figure}
\newpage

\subsection*{Rozdělení}
\begin{itemize}
    \item Číslicobé osciloskopy DSO(Digital Storage Oscilloscope)
    \begin{figure}[H]
        \includegraphics*[scale = 1]{images/osciloskopSchemaDSO.png}
    \end{figure}
    \item Dosvitové osciloskopy DPO(Digital Phosphor Oscilloscope)
    \begin{figure}[H]
        \includegraphics*[scale = 1]{images/osciloskopSchemaDPO.png}
    \end{figure}
    text1 : Snapshosts of the Digital Phosphor contents are periodically sent directly to the display without stopping the acquisition.\\
    text2 : Waveform math, measurements, and front panel control are executed by the microprocessor parallel to the integrated acquisition/display system.\\
    \item MSO - Mixed signal oscilloscope - mix oscloskopu a logického analyzátoru, 2-4 analogové kanály a větší počet digitálních 
\end{itemize}

\subsection*{Výhody oproti analogovým}
\begin{itemize}
    \item lze sledovat signál i před výskytem synchronizace
    \item kvalitnější zobrazení
    \item kvalitnější sledování jednorázových dějů
    \item možnost úpravy signálu, odstranění rušení, operace se signálem
    \item přímé měřená vybraných parametrů, ukládání do paměti a možnost vyvolání paměti
\end{itemize}

\subsection*{Nevýhody oproti analogovým}
\begin{itemize}
    \item výskyt aliasing efektu
    \item omezená přesnost A/D převodníku
    \item složitější obsluha
    \item Cena 
\end{itemize}

\subsection*{Použití}
\begin{itemize}
    \item Analýza časového průběhu
    \begin{itemize}
        \item Měření napětí, napěťových rozdílů, špičkové hodnoty
        \item Měření časových rozdílů, periody, frekvence, doby náběhu a pokles, doby překmitu, šířky pulzu
        \item Měření fázového rozdílu dvou průběhů napětí(Lissajousovy obrazce)
        \item Měření V-A charakteristik elektronických prvků
        \item Měření dynamické hysterezní křivky feromagnetických materiálů
        \item Měření pomocí kurzorů - manuální umístění X a Y, osciloskp určí rozdíl mezi nima
        \item Automatické měření signálu - z nabídky se vyberou požadované parametry, které chcem měřit(některé měří, některé dopočítává)
    \end{itemize}
\end{itemize}

\subsection*{Vlastnosti}
\begin{itemize}
    \item Počet kanálů - dán počtem současně zobrazovaných průběhů
    \item vzorkovací frekvence 
    \item Vstupní napěťové rozsahy - určeny rozsahem vychylovacích činitelů vertikálních kanálů(obvykle 1mV/dílek až 5V/dílek)
    \item Vstupní impedance - obvykle paralelní kombinace odporu 1 MOhm a kapacity 5-50 pF
    \item Přesnost - obvykle 1-5\%
\end{itemize}

\subsection*{Šířka pásma}
Označuje kmitočtové pásmo osciloskopu, ve kterém může být zobrazen harmonický průběh.\\
Všechny osciloskopy mají charakteristiku dolní propust.\\
Šířka pásma odpovídá frekvenci, při které dojde k poklesu o 3 dB(zhruba 30\% chyby měření amplitudy).\\

\subsection*{Vzorkování}
V reálném čase RTS - dostatečný počet vzorků(12-16 během jedné periody).\\
Musí platit vzorkovací teorém.\\
Reži  pretriggering - záznam probíhá neustále a detekce spouštěcí podmínky jej po přednastavené podmínce ukončí.\\
Výhoda co nejvyšší vzorkovací rychlost a co největš paměť.\\
\subsubsection*{Opakované vzorkování}
Dostatečný počet vzorků se sbírá několik period.\\
Menší nároky na HW, protože stačí nižší vzorkovací frekcence.\\
Pouze pro periodické singály.\\
Dvě varianty:
\begin{itemize}
    \item Sekvenční, kogerentní - neumožňuje pretriggering, může trvat déle, vzorky se zobrazují v pořadí zachycení.
    \item Náhodné - umožňuje pretriggering, nutnost ukládat informaci o pořadí vzorku.
\end{itemize}
\begin{figure}[H]
    \includegraphics*[scale = 0.8]{images/opakovaneVzorkovani.png}
\end{figure}
\subsection*{Blokové schéma číslicového osciloskopu}
\begin{figure}[H]
    \includegraphics*[scale = 1]{images/osciloskopSchema2.png}
\end{figure}

\subsubsection*{Vstupní obvody - vertikální kanál}
Přizpůsobení měřeného signálu vnitřním obvodům.\\
Vstupní impedance - typicky 1 MOhm, paralelně 5-50pF.\\
Vazba - DC,AC, GND

\subsubsection*{Blok preprocessingu}
Umožňuje předzpracovat navzorkované data, cyklicky ukládat data bez ohledu na synchronizaci, zvýšit vzorkovací kmitočet. \\
\subsubsection*{Paměť}
FIFO\\
Velikost paměti vzorků má vliv na vzorkovací kmitočet:
\begin{equation}
    f_{vz} = \frac{N}{t \cdot a} 
\end{equation}
$N$ \dots kapacita paměti(počet vzorků)\\
$t$ \dots nastavení časové základny\\
$a$ \dots zobrazení, počet dílků na vodotovné ose\\

\subsubsection*{Synchronizační obvody}
Rozšířené možnosti oproti analogovým osciloskopům.\\
Spouštění úrovní nebo hranou.\\
Spouštění podle délky impulsu.\\
Spouštění počtem impulsů.\\

\subsubsection*{Zobrazení}
Vakuová obrazovka nebo jiný typ(LCD).\\
Nejsou zobrazeny pouze jednotlivé body, meziéhlé hodnoty početně interpolovány.\\
Rastr pro odečítání, kurzory.\\
Pomocné informace o režimu přístroje.\\
Výstup na tiskárnu pro grafickou dokumentaci.\\

\subsection*{Režim sběru dat}
\begin{figure}[H]
    \includegraphics*[scale = 1.2]{images/rezimSberuDat.png}
\end{figure}

\subsection*{Triggering}
\subsubsection*{Spouštění z jednoho zdroje signálu}
Zdroj spouštění signálu:
\begin{itemize}
    \item Interní - spouštění je odvozeno z měřeného průběhu 
    \item Externí - externí signál přivedený na speciální vstup 
\end{itemize}
Způsob spouštění:
\begin{itemize}
    \item Normální - čeká na spouštěcí podmínku
    \item Automatické - vzorkuje stále
    \item Jednorázové - čeká na spouštěcí podmínku a zobrazí pouze jeden průběh 
\end{itemize}
Podmínky spouštění:
\begin{itemize}
    \item Spouštění od hrany a úrovně
    \item Spouštění s nastavitelným zpožděním spouštěcího pulzu - hold-off time
\end{itemize}

\subsubsection*{Spouštění z několika zdrojů}
Logické spouštění množinou logických stavů - kombinace stavů na vybraných kanálech.\\

\subsubsection*{Maska}
Zachycení poruchového "mimotolerantního" signálu.\\
Toleranční pásmo je dané maskou, které nastavíme manuálně toleranci.\\

\subsubsection*{Speciální typy spouštění}
Video signál\\
Pro digitální datové signály v komunikacích.\\

\subsection*{Příslušenství}
\subsubsection*{Osciloskopické sondy}
Používají se ke snímání signálu, základní úpravu jeho velikosti a přizpůsobení k nesymetrickému napěťovému vstupu osciloskopu. \\
Existuje velké množství různých sond pro různé použití.\\
Nejčastější sondy: pasivní napěťové sondy bez útlumu 1:1 nebo s útlumem 1:10\\
Pasivní napěťová sonda:
\begin{itemize}
    \item Kmitočtově kompenzovaný vstupní dělič 
    \item Paralelní kombinace R, C 
    \item Dělící poměr je potřeba brát v úvahu při odečítání napětí z obrazovky osciloskopu, zkreslení měřeného průběhu při špatně vykompenzované sondě
    \item Kompenzace pasivní napěťové sondy
    \begin{itemize}
        \item Slouží k přesnému nastavení reálného přenosu sondy 
        \item Obvykle se využívá obdelníkový signál s rychlými hranami
        \item Kompenzace se provádí nastavovacím prvkem sondy
    \end{itemize}
\end{itemize}

\begin{figure}[H]
    \includegraphics*[scale = 1]{images/osciloskopSonda.png}
\end{figure}

Aktivní napěťová sonda:
\begin{itemize}
    \item obsahuje předzesilovač a je určena pro měření velmi malých signálů
    \item aktivní sonda s nesymetrickým vstupem, nebo pro měření neuzemněných zdrojů signálů 
    \item aktivní diferenciální sonda
    \item víceméně pasivní sonda + zabudovaný zesilovač s velmi nízkou kapacitou
    \item nutnost napájecího zdroje a vyšší cena
\end{itemize}

Proudová sonda:
\begin{itemize}
    \item používá se k zobrazení průběhu proudu na stínítku osciloskopu 
    \item převodníky proudu na napětí 
    \item Pasivní sonda - malé měřicí transformátory proudu 
    \item Aktivní sonda - zpětnovazební sondy s Hallovým generátorem
\end{itemize}

\subsubsection*{Scopemetr}
Měřicí přístroj, který v sobě slučuje funkci osciloskopu, multimetru a bezpapírového zapisovače.\\
Vyrobeno za účelem osciloskopu, které vydrží velmi tvrdé podmínky při práci na místě.\\
Přístroj s LCD displejem, bateriovým napájením a odolnou skříní.\\


\section{Měření frekvence, časového intervalu a fáze. Univerzální čítač - princip, vlastnosti.}
Frekvence a čas jsou definovýny: $f = 1/T$\\
Nejpřesnější měření času a frekvence v porovnání s měřením jiných fyzikálních veličin(etalon - atomové hodiny $10^(-12)$)\\

\subsection*{Měření časového intervalu}
Během měřeného intervalu se čítačem načítávají impulsy dostatečně velkého a stabilního kmitočtu\\
Časový interval vymezen dvojicí krátkých impulsů START-STOP - odvozeny od nástupných/sestupných hran\\
U většiny univerzálních čítačů si můžeme sami aktivní hranu zvolit\\

\subsubsection*{Způsob generování START a STOP impulsu}
Využití dvou vstupních kanálů - na jeden se přivede START impuls a na druhý STOP\\
Využití jednoho vstupního kanálu - oba impulsy odvozovány od náběžné nebo sestupné hrany 

\subsubsection*{Měření časových parametrů signálu}
Délka pulsu\\
Perioda\\
Vzájemný posun dvou signálů\\

\subsubsection*{Časové intervaly lze měřit}
Osciloskopem\\
Číslicovou čítací metodou - univerzální čítač\\
\newpage

\subsubsection*{Způsoby vymezení časového intervalu}
Nutnost vhodného způsobu vymezení začátku a konce.\\
Nejběžnější způsoby:
\begin{figure}[H]
    \includegraphics*[scale = 1]{images/vymezeniCasovehoIntervalu.png}
\end{figure}

\subsubsection*{Číslicové měření časového intervalu}
Přesnost je dána stabilitou a přesností oscilátoru, chyba kvantování času\\
Za pomocí hradla, měří se čas otevření Tx\\
\begin{figure}[H]
    \includegraphics*[scale = 1]{images/minKvantChyb.png}    
\end{figure}

\begin{figure}[H]
    \includegraphics*[scale = 1]{images/minKvantChyb2.png}
\end{figure}

Odvození impulsů START a STOP ze vstupního signálu.\\
Odstranění stejnosměrné složky vstupního signálu kapacitou C(vypínač vypnutý).\\
Zmenšení vstupního signálu na děliči(10x, 100x)\\
Odfiltrování rušivých signálů - filtr dolní propust.\\
Komparátor mění signál na obdelníkový.\\

\subsection*{Univerzální čítač}
\begin{figure}[H]
    \includegraphics*[scale = 1]{images/univerzalniCitac.png}
\end{figure}
\subsubsection*{Přepínatelný dělič frekvence}
Dekadické dělení.\\
Změna časového intervalu.\\

\subsubsection*{Čítač}
Sekvenční logický obvod \\
Stav závisí na počtu přivedených impulsů a odpovvídá použitému kódu(binární, BCD(binary code decimal)).\\

\subsubsection*{Zobrazovač}
Numerický typ, vybaven pamětí\\
Výpočty matematických statistik\\

\subsection*{Měření periody signálu}
Shodné s měřením časového intervalu.\\
START a STOP je odvozen od jednoho signálu.\\

\subsection*{Měření kmitočtu}

\subsection*{Analogové metody}
Využívají selektivní LC nebo RC obvody - mostová nebo rezonanční metoda.\\
Porovnávací metoda\\
Metoda s přímým údajem\\
\newpage 

\subsubsection*{Selektivní obvody}
Vstupní napětí je po případném zesílení přivedeno na selektivní obvod.\\
Výstup z tohoto obvodu přichází na střídavý analogový indikátor(střídavý voltmetr s potřebnou citlivostí).\\
\begin{figure}[H]
    \includegraphics*[scale = 1.3]{images/selekivniObvod.png}
\end{figure}

RC obvod:
\begin{figure}[H]
    \includegraphics*[scale = 1.3]{images/RCObvod.png}
\end{figure}
Charakter frekvenční zátěže
\begin{itemize}
    \item Wienův můstek - mostová metoda
    \begin{figure}[H]
        \includegraphics*[scale = 1.2]{images/wienuvMustek.png}
    \end{figure}
    \item Dvojité T-články, přemostěné T-články
\end{itemize}
Zadržovaná frekvence je nastavitelná:
\begin{itemize}
    \item snaha nastavit tak, aby údaj na indikátoru byl minimální
    \item hodnota frekvence dána parametry zátěže 
\end{itemize}
Nevýhoda - obvod vyvážen jen pro jeden kmitočet, signály s vyššími harmonickýmiznemožňují přesné vyvážení.\\
\newpage

LC obvod:
\begin{figure}[H]
    \includegraphics*[scale = 1.3]{images/LCObvod.png}
\end{figure}
Do 100 MHz, nad = rezonance souosého vedení zakončeného stavitelným zkratem - měření vlnové délky.\\
Snaha nastavit indikátor na maximální hodnotu.\\
\begin{figure}[H]%htbp
    \centering
    \includegraphics*[scale = 1.2]{images/indikatorMax.png}
\end{figure}

\subsubsection*{Porovnavaci metody}
Vyžadují pro svoji činnost zdroj harmoniického signálu se známým a stavitelným kmitočtem.\\
Porovnání kmitočtů:
\begin{itemize}
    \item Směšování - vysoká přesnost
    \item Pomocí dvoukanálového osciloskopu
    \begin{itemize}
        \item Harmonické - Lissajousovy obrazce 
        \item Harmonické i neharmonické - srovnávací metoda
    \end{itemize}
\end{itemize}

Osciloskopická metoda - Lissajousova :\\
\begin{itemize}
    \item změnou kmitočtu nastavíme stojící Lissajousův obrazec, jehož tvar závisí na poměru kmitočtů a jejich fázi 
    \item Snaha o co nejjednodušší obrazec
\end{itemize}
\begin{figure}[H]
    \includegraphics*[scale = 1]{images/Lissajous1.png}
\end{figure}

Osciloskopická metoda - srovnávací : \\
\begin{itemize}
    \item využití dvoukanálového osciloskopu, měřený porovnáváme se známým
    \item synchronizaceod prvního kanálu 
    \item kmitočet měníme, perioda u obou stejná 
\end{itemize}

\subsubsection*{Metoda s přímým údajem}
lineární převodník kmitočtu na stejnosměrné napětí nebo proud\\
existuje ve  formě sondy, kterou lze připojit k univerzálnímu multimetru\\
\begin{figure}[H]
    \includegraphics*[scale = 1.2]{images/primyUdaj.png}
\end{figure}

\subsubsection*{Číslicové měření kmitočtu}

Umožňují i další zpracování naměřených výsledků a mají možnostt připojení k měřicímu systému.\\
Snaha dosáhnout co nejširšího kmitočtového rozsahu, vysoké přesnosti při malé době měření a snadné obsluhy.\\
Nevýhody - u zkreslených nebo zarušených signálů může vést měření k chybným výsledkům.\\

\subsubsection*{Přímé číslicové měření}

Duální měření periody\\
Čítáním se určuje počet period měřeného signálu N, které spadají do měřicího intervalu o známé délce T.\\
\begin{figure}[H]
    \includegraphics*[scale = 1.2]{images/PrimeCislicoveMereni.png}
\end{figure}

\subsubsection*{Nepřímé číslicové měření}
Měření nízkých kmitočtů - výhodnější měřit periodu a kmitočet vypočítat - použití násobiče kmitočtu.\\
Měření vysokých kmitočtů - předřadit rychlý dělič kmitočtu na principu čítače, použití směšovače, kde se získá rozdíl kmitočtů $n.f_p - f_x$ - ten se měří - kmitočtový měnič.\\
Princip:
\begin{itemize}
    \item Vstupní signál přichází na směšovač, kde je ještě přiváděn signál fp z laditelného osciloskopu 
    \item Na výstupu směšovače je signál o kmitočtu n.fp-fx - nastavujeme jej změnou fp tak, aby byl nulový
    \item indikátor musí být selektivní
\end{itemize}
\begin{figure}[H]
    \includegraphics*[scale = 1]{images/NeprimeCislicoveMereni.png}
\end{figure}

\subsection*{Měření fáze}
Fázový posuv $\upvarphi $ vyjádřený ve stupních je možno zjiistit měření časového posunu $t_0$ a periody T (čítačem, osciloskopem). Platí pro harmonické signály.\\
Pro fázový posuv platí:
\begin{equation}
    \upvarphi = \frac{t_0}{T}\cdot 100 \;\;\;\; [\degree]
\end{equation}
\begin{figure}[H]
    \includegraphics*[scale = 1]{images/faze.png}
\end{figure}

\subsubsection*{Metody pro měření fáze}
\begin{itemize}
    \item Kompenzační metoda 
    \item Přímo ukazující fázoměr impulsového typu
    \item Osciloskopické metody 
    \item Metoda s přímým údajem 
    \item Číslicová metoda pro měření okamžité hodnoty fázového rozdílu 
    \item Číslicová metoda pro měření střední hodnoty fázového rozdílu
\end{itemize}

\subsubsection*{Kompenzační metoda}
Obě napětí musí být harmonická.\\
Princip - rozdíl dvou harmonických napětí se stejnou fází, amplitudou a frekvencí je nulový.\\
pomocí zeslabovače a měniče fáze nastavujeme na indikátoru nulové napětí.\\
\begin{figure}[H]
    \includegraphics*[scale = 1]{images/fazeKompenzacni.png}
\end{figure}

\subsubsection*{Osciloskopické metody}
Metoda Lissajousových obrazců:
\begin{itemize}
    \item pouze pro harmonické signály, nebo pro ty, které se od sebe harmonicky liší málo.
    \item Nejpřesnější v oblasti 0  \textdegree až 180 \textdegree , dále přesnost klesá 
    \item Při připojení harmonických napětí nám vznikne "tvar", z jehož rozměrů lze vypočítat fázový rozdíl:
    \begin{equation}
        \upvarphi = arcsin\frac{X_a}{X} = arcsiin\frac{Y_b}{Y}
    \end{equation}
\end{itemize}
\begin{figure}[H]
    \includegraphics*[scale = 1]{images/Lissajous2.png}
\end{figure}

\begin{figure}[H]
    \includegraphics*[scale = 1]{images/Lissajous3.png}
\end{figure}

\subsubsection*{Číslicové měření okamžité hodnoty fázového rozdílu}
Metoda vychází z číslicového měření časových intervalů(viz úvod k měření fáze).\\
Výhody: jsou zachyceny fázové rozdíly za každou periodu\\
Nevýhody: nevhodné pro vyšší kmitočty, protože kmitočet zdroje impulsů by byl příliš velký(volí se zhruba 360x větší, než měřený)
\begin{figure}[H]
    \includegraphics*[scale = 1]{images/fazeCislicovaOkamzita.png}
\end{figure}

\subsubsection*{Číslicová metoda pro měření střední hodnoty fázového rozdílu}
Čítač čítá impulsy ze zdroje impulsů, které prochází dvěma hradly. Hradlo 2 se otvírá na dobu vymezenou průchody nulou měřených signálů $u_1$ a $u_2$.
\begin{figure}[H]
    \includegraphics*[scale = 1]{images/fazeCislicovaStredni.png}
\end{figure}



% SEM psát od zadu
\section{Měření pasivních el. veličin (R,L,C,Z) – metody s přímým údajem, principy mostových metod.}
Pasivní elektrické veličiny popisují vlastnosti pasivních součástek.\\
Definovány pomocí aktivních elektrických veličin.\\
Měřitelné na základě přiloženého napětí na součástce a vyhodnocením velikosti protékajícího proudu součástkou.\\
\subsection{Měření elektrického odporu}
Odpory dělíme na malé (\(<10^{-1}\Omega \)), střední (\(10-10^6\Omega \)) a velké (\(>10^6\Omega \)).\\
Z hlediska měřícího postupu rozeznáváme metody:
\begin{enumerate}
    \item Výchylkové - méně přesné, výsledek ovlivněn přesností jednotlivých členů zapojení a přesností použitých přístrojů.
    \item Nulové metody - přesnost měřícího přístroje neovlivňuje výsledek měření, přístroj se používá pouze k indikaci stavu měření, použití pro přesná měření
\end{enumerate}
\subsubsection{Metody měření elektrického odporu}
\begin{itemize}
    \item Ohmova metoda
    \item Měření odporu uzemněním
    \item Srovnávací metoda
    \item Substituční metoda
    \item Přímo-ukazující měřiče odporu
    \item Nulové metody (mostová měření, číslicové měření odporu)
\end{itemize}

\subsubsection{Ohmova metoda}
Jediná vhodná metoda pro měření lineárních i nelineárních odporů.\\
Hodnota se stanovuje nepřímo výpočtem ze změřeného proudu a napětí.\\
Dvě metody zapojení, obě zatížena chybou metody.\\
\begin{figure}[h!]
    \centering
    \includegraphics[scale = 0.5]{images/Ohm1.png}
\end{figure}
\begin{figure}[h!]
    \centering
    \includegraphics[scale = 0.5]{images/Ohm2.png}
\end{figure}

\subsubsection{Měření odporu uzemněním}
Odpor uzemnění jako přechodový odpor mezi zemnícím vodičem a zemí.\\
Měří se pomocí ohmovy metody.\\
Umístění sond: napěťová na vzdálenost nejméně 20 metrů, proudová nejméně 40 metrů.\\
\begin{figure}[h!]
    \centering
    \includegraphics[scale = 0.5]{images/OdporUzem.png}
\end{figure}

\subsubsection{Srovnávací metoda}
Přenost až 0.1\%.\\
Vhodnost pro měření velkého počtu rezistorů.\\
Snadná realizace automatického měření s využitím jednoho přístroje, který přepíná mezi měřeným a srovnávacím „kusem“.\\
Sériové zapojení pro malé odpory, paralelní pro větší odpory.\\
\begin{figure}[h!]
    \centering
    \includegraphics[scale = 0.5]{images/SrovMetodOdp.png}
\end{figure}

\subsubsection{Substituční metoda}
Slouží pro přesná porovnání skutečných hodnot odporu dvou rezistorů o stejné jmenovité hodnotě.\\
Zvláštní případ porovnávací metody.\\
Použitelné i pro měření jiných veličin (např. kapacita, indukce).\\

\subsubsection{Přímoukazující měřiče}
Analogové. Většinou převodník odpor-napětí.\\
\begin{figure}[h!]
    \centering
    \includegraphics[scale = 0.6]{images/Primouk1.png}
\end{figure}

Nelineární závilost a tudíž nelineární stupnice.\\
Velký měřící rozsah(\(10\Omega - 10M\Omega\)).\\
Konstatní relativní chyba.\\
Změna rozsahu znamená změnu odporu \(R_N\).\\
Převodník odporu na napětí s operačním zesilovačem.\\
Možnost čtyřsvorkového připojení měřeného odporu -> eliminace odporu přívodů a svorkových napětí.
\begin{figure}
    \centering
    \includegraphics[scale = 0.6]{images/Primouk2.png}
\end{figure}

\subsubsection{Nulové metody}
Pro přesné měření odporů (0,01\% laboratorní, 1\% technické).\\
Wheatstoneův můstek:
\begin{itemize}
    \item Nejznámnější můstek k měření odporu
    \item Vhodná pro měření středních odporů
    \item Tvořen 4 rezistory a citlivým nulovým indikátorem
    \item Napájení ze zdroje napětí, nebo proudu
    \item Největší využití při měření neelektrických veličin a v regulační technice
    \item Citlivost Wheatstoneova můstku:
          \begin{itemize}
              \item Do návrhu musí být zahrnut galvanometr
              \item Citlivost vyjádřena jako poměr změn výstupní veličiny a vstupní veličiny
              \item Citlivost závisí na citlivosti galvanometru, napětí zdroje, vzájemném poměru odporů v můstku
          \end{itemize}
    \item Přesnost můstku:
          \begin{itemize}
              \item Závisí na citlivosti můstku, stálosti nulové výchylky galvanometru, přesnosti rezistorů můstku(chyby se sčítají)
              \item Pokud chceme dosáhnout lepší přesnosti, je třeba omezit přechodové odpory a termoelektrické napětí
          \end{itemize}
\end{itemize}
\begin{figure}[h!]
    \centering
    \includegraphics[scale = 0.3]{images/Wheatstone.png}
\end{figure}

Thompsonův můstek:
\begin{itemize}
    \item Určen pro měření malých odporů
    \item Zapojení odporů do obvodu čtyřsvorkově
    \item Přesnost až 0,1\%
\end{itemize}
\begin{figure}[h!]
    \centering
    \includegraphics[scale = 0.4]{images/Thompson.png}
\end{figure}

Číslicové měření odporu - napájení DC nebo AC signálem o nízké frekvenci.\\
\begin{figure}[h!]
    \centering
    \includegraphics[scale = 0.5]{images/CislMerOdp.png}
\end{figure}

\subsection{Měření kapacity}
Kapacita je hlavním parametrem kapacitorů.\\
Kapacitance \(X_C\) je zdánlivý odpor součástky s kapacitou proti průchodu elektrického proudu.\\
Kapacitory pokládáme za lineární a natolik kvalitní, že není důvod měřit parazitní elektrické veličiny (odpor, indukčnost).\\
\subsubsection{Metody měření}
\begin{enumerate}
    \item Přímoukazující přístroje
    \item Rezonanční přístroje
    \item Mostové přístroje
\end{enumerate}
\subsubsection{Přímoukazující měřiče}
\begin{center}
    \(C = \frac{I}{2\pi f U_c}\)
\end{center}
Hodnota kapacity je udávána přímo analogově nebo číslicově pomocí:
\begin{itemize}
    \item Poměru velikostí harmonického proudu I tekoucího měřenou kapacitou C a velikostí napětí na ní
    \item Převodem kapacity na harmonické napětí měřené elektronickým voltmetrem
    \item Převodem kapacity na časový interval měřený číslicově
\end{itemize}
\begin{figure}
    \centering
    \includegraphics[scale = 0.5]{images/PrimoukC1.png}
\end{figure}
Měření kapacity s převodem na napětí:
\begin{itemize}
    \item rezonanční obvod s měřenou kapacitou
    \item indikátor reagující na napětí v rezonančním obvodu (sériový - napěťové buzení, paralelní - proudové)
\end{itemize}
\begin{figure}[h!]
    \centering
    \includegraphics[scale = 0.5]{images/PrimoukC2.png}
\end{figure}
\begin{figure}[h!]
    \centering
    \includegraphics[scale = 0.5]{images/PrimoukC3.png}
\end{figure}

\subsubsection{Mostové měřiče}
Široký měřící rozsah a dobrá přesnost při vyvážení mostu do pravé rovnováhy.\\
Měřič obsahuje:\\
\begin{itemize}
    \item zdroj harmonického signálu
    \item měřící obvod v mostovém zapojení s měřenou kapacitou
    \item střídavý indikátor
\end{itemize}
Vyvážení mostu do rovnováhy:\\
\begin{itemize}
    \item indikátor ukazuje minimální hodnotu výstupního signálu mostu
    \item hodnota měřené kapacity se určí z nastavení vyvažovacích prvků v mostě
\end{itemize}
Rozsah měření 0,1pF - 1mF.\\
\begin{figure}
    \centering
    \includegraphics[scale = 0.5]{images/CMost.png}
\end{figure}

\subsection{Měření indukčnosti}
Duální k měření kapacity, obdobné metody.\\
Indukčnost jako závislost na protékajícím proudu se musí měřit pomocí dostatečně malého harmonického signálu.\\
\subsubsection{Metody měření}
\begin{itemize}
    \item Přímo ukazující
    \item Rezonanční
    \item Mostové
\end{itemize}

\subsubsection{Rezonanční měřič}
\begin{figure}[h!]
    \centering
    \includegraphics[scale = 0.5]{images/RezonMerL.png}
\end{figure}

\subsubsection{Mostový měřič}
\begin{figure}[h!]
    \centering
    \includegraphics[scale = 0.5]{images/MostMerL1.png}
\end{figure}

\begin{figure}[h!]
    \centering
    \includegraphics[scale = 0.5]{images/MostMerL2.png}
\end{figure}

\subsubsection{Měření vzájemné indukčnosti}
Vzájemná indukčnost vyjadřuje stupeň vazby mezi dvěma vázanými induktory.\\
Metody měření: Přímá, diferenční, měření naprázdno a nakrátko.\\

Přímá metoda:\\
\begin{figure}[h!]
    \centering
    \includegraphics[scale = 0.5]{images/VzLPrima.png}
\end{figure}

Diferenční metoda:\\
\begin{figure}[h!]
    \centering
    \includegraphics[scale = 0.5]{images/VzLDiff.png}
\end{figure}
\newpage
Měření naprázdno a nakrátko:\\
\begin{figure}[h!]
    \centering
    \includegraphics[scale = 0.5]{images/VzLNN.png}
\end{figure}

\subsection{Měření impedance(imitance)}
Imitance je společný název pro impedanci(zdánlivý odpor) a admitanci(zdánlivá vodivost).\\

\subsubsection{Metody měření}
\begin{itemize}
    \item Metody s přímým měřením
          \begin{itemize}
              \item R, L, C
              \item Z, Y
              \item Analyzátory obvodů
          \end{itemize}
    \item Rezonační obvody
          \begin{itemize}
              \item Sériová rezonance, Q-metry
              \item Paralelní rezonance, měřiče ztrátového odporu
          \end{itemize}
    \item Nulové metody (nejpřesnější)
          \begin{itemize}
              \item Mostové
              \item Kompenzační
          \end{itemize}
    \item Pro VVF využívající (nad 100Mhz)
          \begin{itemize}
              \item Měřící vedení(náročné mechanicky a konstrukčně)
              \item Směrové odbočnice
          \end{itemize}
\end{itemize}

\subsubsection{Měření imitance}
Určení imitance měřením napětí a fáze. Na základě ohmova zákona z hodnot harmonického napětí na imitanci a proudu protékajícím imitancí. Modul impedance Z se určí z poměru amplitud obou veličin. Fázový úhel se určí z fázového rozdílu těchto veličin.\\
Dvy typy zapojení, buzení s konstantním napětím a buzení s konstantním proudem.\\

\subsubsection{Měřič imitance s přímým údajem}
Neznámá impedance Zx je vypočtena z měřeného napětí a proudu.\\
\begin{figure}[h!]
    \centering
    \includegraphics[scale = 0.4]{images/ImitPrim.png}
\end{figure}

\subsubsection{IV sonda}
Vychází z měření přímého údaje a jeho výpočtu impedance.\\
Proud je u sondy měřen nepřímo přes vysokofrekvenční proudový transformátor.\\
Základní metoda do 100 MHz.\\

\subsubsection{Q metr}
Vychází z rezonanční metody - sériové zapojení.\\
Slouží pro stanovení činitele jakosti Q.\\

\subsubsection{Rezonanční metoda}
Měření vychází z chování rezonančních obvodů ve stavu resonance.\\
Možnosti:\\
\begin{itemize}
    \item Paralelní rezonanční obvod - hledáme maximum napětí na rezonančním obvodu při napájení ze zdroje s velkým vnitřním odporem
    \item Sériový rezonanční obvod - hledáme maximum proudu s malým vnitřním odporem
\end{itemize}
Modifikace metody vychází z toho, zda je k dispozici normálová kapacita nebo indukčnost.\\

\subsubsection{Mostová měření}
Do 100 MHz. \\
Mechanicky náročné.\\
Přesnost 1\%.\\
Typy mostů:
\begin{itemize}
    \item Pro imitanci indukčního charakteru:
          \begin{itemize}
              \item Owanův
              \item Hayův
              \item Maxwell-Wienův
              \item Rezonanční
          \end{itemize}
    \item Pro imitanci kapacitního charakteru:
          \begin{itemize}
              \item De Sautyho
              \item Scheringův
          \end{itemize}
\end{itemize}

\subsubsection{Srovnání metod měření impedance}

\begin{figure}[h!]
    \centering
    \includegraphics[scale = 0.4]{images/srovnani.png}
\end{figure}

\section{Měření magnetických veličin. Snímače magnetických veličin (princip základních magnetických převodníků - měřicí cívka, Hallova sonda). Měření parametrů feromagnetik (hysterezní smyčka, ztráty) pomocí osciloskopu.}
Problémy při měření magnetických veličin:
\begin{itemize}
    \item Obtížné stanovení přesné dráhy magnetického toku a plochy, kterou magnetické toky prostupují.
    \item Změna teploty vzorku ovlivňuje magnetické vlastnosti.
    \item Mechanické namáhání mlže ovlivnit magnetické vlastnosti.
    \item Ovlivnění výsledků měření vnějšími magnetickými poli.
\end{itemize}

\subsection{Magnetické převodníky}
Převod měřené veličiny (B, H, \(\mu r\)) na elektrický signál.\\
Typy převodníků:
\begin{itemize}
    \item Měřící cívka
    \item Hallova sonda
    \item Feromagnetická sonda
    \item Rogowskiho-Chattockův potenciometr
    \item Anizotropní magnetorezistor
\end{itemize}
\subsection{Měřící cívka}
Převodník změny magnetického toku \(\Phi \) [Wb] na elektrické napětí u(t).\\
Použití pro střídavé magnetické pole bet stejnosměrné složky.\\
\begin{center}
    \(u(t) = -N\cdot\frac{d\Phi}{dt} = -N\cdot S\cdot \frac{dB}{dt}\)
\end{center}
Kmitočtové omezení je dáno vlastní rezonancí cívky.\\
Mění-li se magnetický tok \(\Phi\) obepínaný N závity měřicí cívky, indukuje se v cívce napětí u(t):
\begin{figure}[h!]
    \centering
    \includegraphics[scale = 0.4]{images/MerCivka.png}
\end{figure}

\subsection{Hallova sonda}
Využití Hallova jevu.\\
Sonda s polovodičovou destičkou.\\
Stejnosměrný proud I tekoucí v podélném směru.\\
Lorentzova síla - působí na pohybující se nosiče elektrického náboje v magnetickém poli. Jev spočívá ve vychylování směru toku el. Proudu v závislosti na velikosti indukce mag. Pole, které je kolmé na polovodičovou destičku.\\
\begin{figure}[h!]
    \centering
    \includegraphics[scale = 0.5]{images/MVEHall1.png}
\end{figure}
\begin{center}
    \(u_H = \frac{R_H\cdot I}{d}\cdot B [V]\)
\end{center}
Kde \(R_H\) je Hallova konstanta \([m^3/C]\)

\begin{figure}[h!]
    \centering
    \includegraphics[scale = 0.5]{images/MVEHall3.png}
\end{figure}

\subsubsection{Charakteristika a použití}
\begin{itemize}
    \item Malé konsrukční rozměry
    \item Měření kolmé složky magnetické indukce ve vzduchových mezerách
    \item Měření nejsou ovlivňována, sonda nemá feromagnetické části
    \item Malý napájecí proud(10-100mA)
    \item Dobrá linearita převodní charakteristiky
    \item Meření stejnosměrných a střídavých magnetických polí nízkých kmitočtů
    \item Nevýhoda - teplotní závislost (kompenzace pomocí polovodičů s opačným teplotním koeficientem než má sonda)
\end{itemize}

\subsection{Feromagnetická sonda}
Nejrozšířenější typ magnetické sondy obsahuje 2 jádra. Má 2 feromagnetické jádra z kvalitního feromagnetického materiálu(měkký ferit) a ta jsou ovinuta stejným počtem závitů a umístěna do vnější cívky. Vinutí jader jsou zapojena proti sobě a napájena střídavým proudem.\\
Vlivem vnějšího stejnosměrného magnetického pole s intenzitou H, je souměrnost magnetický toků \(\Phi_1\) a \(\Phi_2\) porušena a na vnější cívce se indukuje napětí úměrné H.\\
\begin{figure}[h!]
    \centering
    \includegraphics[scale = 0.5]{images/FeromagSonda.png}
\end{figure}
\subsubsection{Charakteristika a použití}
\begin{itemize}
    \item Měření velmi slabých magnetických polí - stejnosměrné nebo nízkofrekvenční střídavá pole.
    \item Měřící rozsah \(100pT\) až \(200\mu T\)
    \item Vysoká citlivost oproti Hallově sondě
    \item Měření kolísání magnetického pole země
    \item Magnetické kompasy v letadlech a lodích
    \item Hledání rudných ložisek a feromagnetických materiálů pod povrchem
    \item Indikace vad materiálů
\end{itemize}

\subsection{Rogowskiho-Chattockův potenciometr}
Potenciometr jako zvláštní úprava měřící cívky pro integrační měření.\\
Pásek z nemagnetického a nevodivého materiálu s konstantním průřezem rovnoměrně ovinutý po celé délce vodičem.\\
Změny magnetického toku \(\Phi\) v cívce -> vznik indukovaného impulsu napětí u(t).\\
Konstrukce: Měřící cívka pro integrační měření. Měření změn magnetického napětí \(U_m\).\\
\begin{figure}[h!]
    \centering
    \includegraphics[scale = 0.5]{images/RChPoten.png}
\end{figure}

\subsection{Anizotropní magnetorezistor(AMR)}
Magnetorezistivita - změna odporu s magnetickým polem.\\
Anizotropní magnetorezistivita - změna odporu souvisí se změnou úhlu mezi vektorem magnetizace a směrem proudu, který protéká daným materiálem.\\
Princip: Změna elektrického odporu vlivem magnetického pole - dochází k ovlivňování nosičů náboje, což se projeví jako změna odporu materiálu. Vodivost feromagnetika ve směru magnetizace je menší než ve směru kolmém.\\
AMR efekt: elektrický opor ve směru magnetizace je vyšší než ve směru kolmém.\\
Provedení: tenká vrstva permalloye je nanesena na křemíkový substrát.\\
Remanentní magnetizace leží ve směru x (remanentní magnetizace je zbytková magnetizace, kterou si udrží feromagnetický materiál, když na něj přestane působit magnetické pole).\\
\begin{figure}[h!]
    \centering
    \includegraphics[scale = 0.5]{images/AMR.png}
\end{figure}

\(H_Y\) otáčí magnetizaci M a velikost R se mění o 2-3\%.\\
Senzor je tvořen tenkou vrstvou slitiny železa a niklu (permalloy). Pokud na senzor nepůsobí magnetické pole, má svůj klidový odpor. Působením magnetického pole tento klidový odpor klesá téměř lineárně(jen v rozsahu 2-3\%).\\
Použití:
\begin{itemize}
    \item Stejnosměrné i střídavé pole
    \item Měřící rozsah B od \(1\mu T\) do \(10mT\), v můstkovém uspořádání \(10nT\) až \(100\mu T\).
    \item Frekvenční rozsah DC až MHz
\end{itemize}
Vlastnosti: Závilost odporu na úhlu mezi proudem a magnetizací je silně nelineární.\\
Linearizace pomocí Barber poles struktury: Na pásek z permalloye jsou napařeny pod úhlem 45°proužky hliníku, který má výrazně vyšší vodivost než permalloy. Proud teče mezi hliníkovými proužky pod úhlem 45° k ose proužku, tím dochází k posunu charakteristiky. Tato technika vhodnější pro menší pole (kompas). \\
\begin{figure}[h!]
    \centering
    \includegraphics[scale = 0.7]{images/AMRLin.png}
\end{figure}

\subsection{Měření magnetizačních charakteristik materiálů}
\subsubsection{Vzorky}
Uzavřené:
\begin{itemize}
    \item Tvoří homogenní magnetický obvod s konstantním průřezem nepřerušený vzduchovou mezerou nebo s mezerou vyplněnou jiným materiálem
    \item Magnetický tok se uzavírá pouze feromagnetikem
    \item Výhoda - intenzita magnetického pole lze vypočítat z magnetizačního proudu, počtu závitů magnetizačního vinutí a střední délky siločar
\end{itemize}

Otevřené:
\begin{itemize}
    \item Obvykle tvar tyčí nebo pásků
    \item Magnetický tok se uzavírá vně vzorku vzduchem
    \item Intenzita magnetického pole se nedá určit výpočtem
    \item Obtížné dosažení homogenního rozložení toku v celém objemu vzorku
\end{itemize}
\begin{figure}[h!]
    \centering
    \includegraphics[scale = 0.6]{images/Vzorky.png}
\end{figure}

\subsubsection{Křivky}
Křivka prvotní magnetizace:
\begin{itemize}
    \item Výchozí stav - dokonalé odmagnetování
    \item Pomalu zvyšující se intenzita mag. Pole jedním směrem
    \item Nesmí nastat změna intenzity H opačného směru
\end{itemize}
Hysterezní křivka:
\begin{itemize}
    \item Vyjadřující závislost B = f(H) pro magnetizační cykly
    \item Změny intenzity mag. Pole → +H … -H
\end{itemize}
\begin{figure}[h!]
    \centering
    \includegraphics[scale = 0.5]{images/KrivkyMag.png}
\end{figure}ˇ

\subsubsection{Měření na uzavřených vzorcích}
\begin{figure}[h!]
    \centering
    \includegraphics[scale = 0.5]{images/MerUzavVzor.png}
\end{figure}
Elektronický Wb-metr:
\begin{itemize}
    \item Tvořen integračním zesilovačem
    \item Nepříznicvě působí rušivá termoelektrická napětí ve vstupním obvodu. Kvůli tomu se používají vodiče a kontakty s malým termoelektrickým napětím.
    \item Rozsahy lze měnit přepínáním velikosti integračního rezistoru
\end{itemize}
\begin{figure}
    \centering
    \includegraphics[scale = 0.9]{images/Wb-metr.png}
\end{figure}

\subsubsection{Měření na otevřených vzorcích}
Otevřené vzorky mají tvar válečku nebo kostky, jsou k dispozici častěji než prstencové vzorky.\\
Problém je správné určení intenzity magnetického pole, nelze ji odvodit z proudu jako u uzavřených vzorků -> pro magnetování se používá jha(ruční elektromagnet) z magneticky měkkého materiálu.\\
Postup:\\
\begin{itemize}
    \item Magnetické pole je buzeno magnetovacím vinutím \(N_1\) napájeným magnetovacím proudem ze zdroje Z.
    \item Tangenciální Hallova sonda (HS) slouží při určení intenzity H
    \item Magnetická indukce B se měří měřící cívkou s \(N_2\) závity pomocí fluxmetru F
\end{itemize}
Hysterezigrafy - přístroje jsou řízeny počítačem a proces měření je digitalizován.\\

\begin{figure}[h!]
    \centering
    \includegraphics[scale = 0.5]{images/MerOtevVzor.png}
\end{figure}