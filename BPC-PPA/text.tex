\section{Úrovně řízení výroby a jejich funkce. Zařazení komponent do jednotlivých vrstev a možnosti jejich propojení. Způsob řízení výroby (centralizované a distribuované). Toky dat (informací) v systému a jejich popis. Vlastnosti a možnosti nadřazených výrobních systémů (MES, ERP).}
\subsection{Struktura řídícího systému výrobního podniku}
\begin{figure}[h]
    \begin{center}
      \includegraphics[width=\textwidth]{img/rizeni.png}
    \end{center}
  \end{figure}

\subsubsection*{Úrověň řízení podniku - ERP}
\begin{itemize}
    \item Podnikový informační systém - ERP (= Enterprise Resource Planning).
    \item Zařizuje nákup, logistiku, distribuci, účetnictví, fakturace.
\end{itemize}

\subsubsection*{Úroveň řízení výroby - MES}
\begin{itemize}
    \item Výrobní informační systém - MES (= Matufacturing Execution System).
    \item Shromažďuje data o výrobě a na základě těchto dat posílá příkazy do jednotlivých částí výroby.
    \item Věnuje se například správě výrobních zdrojů, plánováním výroby, řízení výroby, sběr dat, kontrola jakosti, výkonnostní analýzy.
\end{itemize}

\subsubsection*{Operátorská úroveň - SCADA, HMI}
\begin{itemize}
    \item Systém SCADA (= Supervisory Control And Data Acquisition), HMI (= Human Machine Interface) - nadřazená úroveň PLC.
    \item SCADA je vzdálený monitorovací systém - například velín. Jedná se o software monitorující průmyslová a technická zařízení, jejich procesy a umožňuje vzdálené ovládání.
    \item HMI je dotyková obrazovka, umožňující ovládání a monitorování procesu přímo u daného zařízení. 
    \item Lidská činnost - definování procesu, hlídání chybových hlášení, správa atp.
\end{itemize}

\subsubsection*{Úroveň řídícíh buněk}
\begin{itemize}
    \item Bezprostřední řízení procesů pomocí PLC, IPC, regulátory, procesní stanice.
    \item Systémy přímo napojené na výrobní prostředek, spravující jeho sledování a nastavování. 
\end{itemize}

\subsubsection*{Procesní úroveň}
\begin{itemize}
    \item Snímače, akční členy (aktuátory), motory, roboti.
    \item Snímače získávají data z výrobního procesu. Akční členy tyto procesy nějakým způsobem ovlivňují. 
\end{itemize}

\subsubsection*{Technologie}
\begin{itemize}
    \item Elektrické přostředky, tepelné zařízení, dopravníkový pás, převodovky,...
\end{itemize}

\subsection{Možnosti propojení úrovní řízení}
\begin{itemize}
    \item ERP - MES = komunikace pomocí ethernetu.
    \item SCADA - PLC = komunikace pomocí ethernetu - například standard OPC UA.
    \item PLC - HMI = komunikace pomocí sériové komunikace, ethernet. Například Modbus TCP, Ethernet/IP, ProfiNET.
    \item PLC - Snímače = propojeny elektronicky (pomocí digitálních/analogových vstupů/výstupů - proudové smyčky 4-20mA), sběrnicí - IO-Link, AS-Interface, Profibus.
\end{itemize}

\subsection{Centralizované a distribuované řízení výroby}
\subsubsection*{Distribuované - DCS}
\begin{itemize}
    \item DCS = Distributed Control System.
    \item Každá řídící komponenta má na starost svou dílčí oblast, která je do jisté míry autonomní - například PLC.
    \item Tyto komponenty komunikují mezi sebou, případně s nadřazeným systémem, který je ovládá jako celek.
\end{itemize}

\subsubsection*{Centralizované}
\begin{itemize}
    \item Řízení z jednoho organizačního ústředí = jeden řídící komponent, kde se vyhodnocuje logika pro řízení všech procesů.
\end{itemize}

\begin{figure}[h]
    \begin{center}
      \includegraphics[scale = 1]{img/picture2.png}
    \end{center}
  \end{figure}

  \subsection{Toky dat a druhy řízení}
  V principu ERP zapisuje do MES požadavky na výrobu, MES podle požadavku nastavuje výrobní prostředky. Z výrobních prostředků se vrací informace o jejich stavu (výkonnost, opotřebení, spotřeba materiálu), které MES nějakým způsobem zaobaluje, aby tyto informace byly „použitelné pro business“.

  \begin{figure}[h]
    \begin{center}
      \includegraphics[scale = 1]{img/picture3.png}
    \end{center}
  \end{figure}


  \subsection{Vlastnosti a možnosti nadřazených výrobních systémů (MES,ERP)}
  \subsubsection*{ERP}
  \begin{itemize}
    \item Slouží k řízení podniku, obchodní část systémů, logistika, distribuce, správa majetku, faktury, učetnictví,\dots
    \item Výhody:\begin{itemize}
        \item Zefektivnění a zrychlení podnikových procesů
        \item Centralizace a vyčištění dat, snížení chybovosti
        \item Méně byrokracie
        \item Vyšší bezpečnost
    \end{itemize}
    \item Funkce: \begin{itemize}
        \item Vyřizování objednávek
        \item Nákup materiálu
        \item Zajišťování lidských zdrojů
        \item Výpočet ziskovosti
        \item Prodej a distribuce produktů
    \end{itemize}
  \end{itemize}

  \subsection{MES}
  \begin{itemize}
    \item Informační a řídící systém podporující efektivní provádění výrobních operací.
    \item Sbírá aktuální a přesná data, navádí a spouští aktivity v závodě a podává informace "výš".
    \item Funkce: \begin{itemize}
        \item Správa výrobních zdrojů a postupů
        \item Plánování výroby a řízení ze strany dispečera
        \item Jakostní a výkonnostní analýzy
        \item Sběr dat
    \end{itemize}
    \item Spolupráce s ERP: \begin{itemize}
        \item Nastavení výroby dle požadavků ERP - množství, kvalita, receptura = možnosti výroby.
    \end{itemize}
  \end{itemize}

\section{Standardní rozhraní průmyslových signálů - typy, obvodové provedení, vlastnosti a použití. Logika digitálních signálů. Zpracování analogové veličiny. Standardizace a destandardizace. Senzory - popis, typy a jejich použití. Možnosti zapojení
snímačů do systému.}

\subsection{Standardní rozhraní průmyslových signálů - typy, obvodové provedení, vlastnosti a použití}
Rozhraní je fyzicky tvořeno buď tranzistorovou logikou, nebo pomocí relé. Používá se několik napěťových a proudových úrovní.
\subsubsection*{Rozhraní - spojité}
\begin{itemize}
  \item Průmyslová logika 24 VDC: \begin{itemize}
    \item Logická "0" \dots -30 - 5 VDC
    \item Logická "1" \dots 13 - 30 VDC
  \end{itemize}
  \item Značení: \begin{itemize}
    \item NO - Normally Open (pokud není energizován, je rozepnut) = spínací kontakt
    \item NC - Normally Closed (pokud není energizován, je sepnut) = rozpínací kontakt
    \item FO - Fail Open (Při poruše zůstane rozepnut)
    \item FC - Fail Close (Při poruše zůstane sepnut)
    \item FL - Fail Last (Při poruše zůstane v poslední poloze)
  \end{itemize}
  	\item Fyzicky jsou vstupy realizovány jako: \begin{itemize}
      \item HTL (High Treshold Logic) - push-pull - využity dva tranzistory na výstupu, dosah až 100 m, pro prostředí s větším rušením - vhodné pro průmysl
      \item TTL (Transistor Transistor Logic) - logická "1" je pouze 2-5 VDC, využitá se u RS422 (Kroucená dvoulinka, dosah až kilometr), obecně napájeno z nižších napětí (5V, 1,7V), používá se na DPS.
      \item Open Collector - Pouze 1 tranzistor, dosah 10 m. Logická "1": 2-30 VDC, Logická "0": 0 - 0.5 VDC.
    \end{itemize}
  \end{itemize}
  
  \subsubsection*{Obvodové provedení}
  \begin{itemize}
    \item Sink a Source - Udává, na keré straně zdroje se spíná
    \item Sink se připojuje se zemí
    \item Source se spojuje s Ucc
    \item Pro sourcing snímač se použije sinking vstup a naopak 
  \end{itemize}

  \begin{figure}[h]
    \begin{center}
      \includegraphics[scale = 1]{img/picture5.png}
    \end{center}
  \end{figure}

  \subsubsection*{Zpracování analogových veličin}
  \begin{itemize}
    \item Napětí - rozsah: 0 až 10 V, dochází rušení z ostatních vodičů díky indukci, napětí s delkou vedení klesá.
    \item Proud - rozsah 4-20 mA (0-20 mA), tzv. proudová smyčka, odolnější proti rušení, lze detekovat poruchu (0 mA) a ze smyčky lze zařízení i napájet. 
    \item Zapojení může být 2-vodičové (např. měření odporu PT100), 3-vodičové (např. měření polohy (potenciometr)), 4-vodičové (snímač s vlastním napájením + proudová smyčka) 
  \end{itemize}
  
  \begin{figure}[h]
    \begin{center}
      \includegraphics[scale = 1]{img/picture4.png}
    \end{center}
  \end{figure}

  \subsubsection*{Měřící rozsahy}
  \begin{itemize}
    \item Bipolární (kladné i záporné), Unipolární (pouze kladné)
  \end{itemize}

  \subsection{Standardizace a destandardizace}
  \subsubsection*{Standardizace}
  \begin{itemize}
    \item Jedná se o převod elektrické veličiny na reálné číslo v příslušných inženýrských jednotkách, které se budou nacházet v zadaných mezích.
    \item Standardizovaná hodnota lze dále uživatelsky využívat - např. zobrazení na operátorském panelu.
    \item x = naměřená hodnota
    \item MAX,MIN = vstupní analogová proměnná (např. teplota)
    \item k1 = min hodnota převodníku
    \item k2 = max hodnota převodníku 
  \end{itemize}

  \begin{figure}[h]
    \begin{center}
      \includegraphics[scale = 1]{img/picture6.png}
    \end{center}
  \end{figure}

\subsubsection*{Destandardizace}
\begin{itemize}
  \item Opačný převod jako standardizace. 
  \item Převod reálného čísla na elektrickou veličinu.
  \item Využívá se k řízení pomocí akčních regulačních členů.
\end{itemize}

\subsection{Senzory}

\subsubsection*{Čidlo}
\begin{itemize}
  \item Převodník mechanické/tepelné/elektrické/magnetické/... energie na elektrickou.
  \item Jednoduché a rychlé zpracování analogové i digitální.
  \item Rychle měření a spolehlivý přenos.
\end{itemize}

\subsubsection*{Inteligentní snímač}
\begin{itemize}
  \item Měří a rovnou i zpracovává měřenou veličinu.
  \item Obsahuje procesor, má standardní výstup nebo komunikační sběrnici (protokol).
  \item Skládá se z modulu pro měření, vyhodnocení a komunikaci.
\end{itemize}

\subsubsection*{Rozdělení}
\begin{itemize}
  \item Aktivní/pasivní:\begin{itemize}
    \item Aktivní - Chová se jako zdroj elektrické energie
    \item Pasivní - Potřebuje ke své funkci napájení
  \end{itemize}
  \item Relativní/absolutní: \begin{itemize}
    \item Relativní - Výstupní veličina je úměrná měřené (konstanta úměrnosti)
    \item Absolutní - Výstupní veličina přímo odpovídá měřené
  \end{itemize}
  \item Dle typu měřené veličiny: \begin{itemize}
    \item Lineární
    \item Rotační
  \end{itemize}
  \item Dle spojení s prostředím: \begin{itemize}
    \item Kontaktní
    \item Bezkontaktní
  \end{itemize}
  \item Dle typu výstupního signálu: \begin{itemize}
    \item Dvoustavové (binární)
    \item S kódovým výstupem
    \item Inkrementální, absolutní
    \item Se spojitým výstupem
  \end{itemize}
  \item Podle fyzikálního principu: \begin{itemize}
    \item Odporové, kapacitní, indukčnostní, magnetické, piezoelektrické, optoelektronické,\dots
  \end{itemize}
  \item Podle měřené veličiny: \begin{itemize}
    \item Snímače tlaku, průtoku, teploty, radiačních, mechanických, magnetických a elektrických veličin
  \end{itemize}
  \item Podle Výrobní technologie: \begin{itemize}
    \item Mechanické, pneumatické, elektrické, elektromechanické, polovodičové, optoelektronické, mikroelektronické (MEMS), \dots
  \end{itemize}
  \item Podle jiných požadavků: \begin{itemize}
    \item Cena, provozní náklady, spolehlivost, rozměry, hmotnost, přesnost, citlivost, \dots
  \end{itemize}
\end{itemize}

\textbf{!!! Ostatní info k jednotlivým snímačům už určitě znáš z otázek předmětu BPC-SNI ;) !!!
}
