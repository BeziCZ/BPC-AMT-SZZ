\section{Bezpečnost na~vrstvě L1 (bezpečnostní opatření, klíčování, dohled nad~sítí) a bezpečnost na~vrstvě L2 (bezpečnostní opatření, příklady útoků, MACsec).}

Na~fyzické vrstvě pracují technologie jako USB, Bluetooth, UTMS, 100Base-T, DSL, \dots

Na~spojové vrstvě pracují protokoly Ethernet, PPP; WEP/WPA\{,2,3\}; PAP/CHAP/EAP, \dots


\subsection{Bezpečnostní opatření na~L1}

Zajištění a~izolace kabelů nebo optiky, síťových prvků, ochrana fyzických rozhraní a portů:
\begin{itemize}
	\item Zamezení neautorizovaného připojení do~portů.
	\item Zamezení poškození síťového zařízení.
	\item Umístění bezpečnosti perimetru nebo prostoru (čidla, kamery, kontrola přístupu, uzamykatelné skříně).
	\item Pasivní ochrana: blokátory (pro~konektory a porty), klíčování (znemožnění připojení cizích skupin PatchCordů a LC konektorů).
\end{itemize}

Obfuskace komunikace (kódování, přidání bílého šumu).
Analogové šifrování (různé modulace).
Kvantová kryptografie (distribuce klíčů protokolem BB84).

Šifrování na~první úrovni způsobuje problémy (synchronizace, komplexita, neefektivní ošetření chyb).


\subsection{Dohled nad~L1 sítí}

AIM (\emph{Automatized Infrastrucutre Management}) zajišťuje, dokumentuje a~monituruje síťovou kabeláž.
Umožňuje detekovat vložení a odstranění propojovacích kabelů (pomocí mikrospínač nebo například měřením impedančních vlastností).

IPLMS (\emph{Intelligent Physical-Layer Management Solution}) kombinuje inteligentní propojovací panely se softwarovými funkcemi a poskytuje informaceo~stavu připojení na~portech.


\subsection{Rizika a~hrozby na~L1}

Externí útoky: modifikace dat (\emph{tampering}), rušení (\emph{jamming}), odepření služeb (DoS), odposlech (\emph{sniffing}; bezdrátově, metalikou, optikou, prostorově pomocí kamery nebo mikrofonu), obejití přístupu (\emph{authentication bypass}; lámání hesel, obnovení do~továrního nastavení, nouzový režim, reboot), útok postranním kanálem (proudově, napěťově, zvukově, opticky).

Interní útoky: poškození kabelů nebo zařízení, neoprávněné připojení k~portu (\emph{tap}, \emph{splitter}) a odposlech či modifikace dat.


\subsection{Bezpečnostní opatření na~L2}

Ethernet packet obsahuje pouze CRC32, nelze ho považovat za~bezpečnostní prvek.

802.1x: kontrola přístupu zařízení do~sígě na~portech nebo na~bezdrátových přístupových bodech.
Doporučuje se EAP a integrace s~AAA (Radius, Diameter).

802.11: WEP i WPA obsahují mnoho chyb, WPA2 (802.11i): KRACK (2017), WPA3: zranitelnost v~Dragonfly handshake.


\subsection{Rizika a~hrozby na~L2}

DoS (vyčerpání MAC adres, kolize), odposlech, modifikace dat.

Existuje množství útoků: spoofing MAC adres, \emph{ARP cache poisoning}, \emph{ARP cache flooding}, WEP/WPA zranitelnosti, VLAN útoky.

% TODO Příklady útoků


\subsection{MACSec}

802.1ae je specifikace implementace SecY (\emph{MAC Security Entities}).
Zajišťuje důvěrnost dat, autenticitu i~integritu.
Jako takový nezajišťuje management klíčů a ustanovení bezpečné relace; existuje 802.1x-2010 jako MKA (\emph{\mbox{MACSec} Key Agreement}).

802.1ae+802.1x dohromady zajišťují oboustrannou autentizaci, výměnu klíčů, šifrování, integritu a autentičnost.
Jsou součástí Linuxového jádra 4.5+, některých CISCO přepínačů a dalších L2 prvků.

Protože jde o~ochranu na~L2, chrání data jen uvnitř jednoho LAN segmentu a při~přechodu do~jiné sítě je terminován.
Podporuje VLAN.

Rámce jsou podobné Ethernet rámcům, ale obsahují navíc \emph{Security Tag} a MAC.
Používá se AES-GCM-128 (nebo -256).


\clearpage
\section{Bezpečnost na~vrstvě L3 (bezpečnostní opatření, příklady útoků, IPsec, bezpečnost IPv6).}

L3 je první vrstva na~které je zavádění šifrování praktické i pro~běžné použití.

Zajištění autentizace/autorizace komunikujících stran (PKI, AAA, ACL), důvěrnost (VPN), bezpečnost služeb (ICMP, ARP, IGMP), QoS, bezpečnost protokolů (OSPF, RIP).


\subsection{Bezpečnostní opatření na~L3}

Zabezpečení.
Key management (PKI), autentizace (AAA řešení, Radius, Tacacs, Kerberos), ACL, firewally.

Důvěrnost a integrita.
Bezpečnost je založena na~vyšších vrstvách (TLS, DTLS, QUIC).
IPSec.

Ochrana funkčnosti sítě.
Směrovací protokoly nemívají možnost vzájemné autentizace (OSPFv2, RIPv2, EIGRP, BGP: zcela bez autentizace, případně MD5 PSK).


\subsection{Rizika a~hrozby na~L3}

Spoofing, IP flooding, ICMP flooding, Smurf DDoS (podvržení zdroje v~ICMP dotazu), wormhole, blackhole, sybil.

U~dynamických směrovacích protokolů je možné do~sítě nasadit škodlivé zařízení které může síť ovládnout.


\subsection{IPSec}

IPSec poskytuje bezpečnost na~třetí vrstvě včetně managementu klíčů a ověření protistran.
Jde o~sadu několika protokolů: \emph{Authentication Header} (integrita), \emph{Encapsulating Security Payloads} (šifrování), \emph{Security Associations} (parametry spojení).
Lze ho využívat v~tunelovacím (vše šifrované, nová hlavička) nebo transportním (šifrovaná data) módu.
Ustanovení klíče je možné přes PSK, IKE1/IKE2, Kerberos.

Implementován ve~Windows, OSX, nových verzích Android, pro~Linux \{strong,libre,open\}swan.
Alternativou k~IPSec je např. Wireguard.


\subsection{Bezpečnost IPv6}

V~IPv6 měl být IPSec původně vyžadován, dnes jde však jen o~doporučení.
IPv6 používá nové přístupy a~protokoly které zvětšují prostor pro~možné útoky.
Některé útoky z~v4 lze adaptovat: ARP $\rightarrow$ NDP, broadcast $\rightarrow$ multicast multiplikace, fragmentace (směrovače $\rightarrow$ uzly).
Je možné zneužívat rozšířené hlavičky.

Kvůli výrazně většímu prostoru je v6 odolná vůči skenování; 64b prefix, různé způsoby přidělování adres.

% TODO Bezpečnost multicastu



\clearpage
\section{Bezpečnost TCP (útoky, protiopatření), protokol TLS -- Transport Layer Security (princip, součásti, příklady útoků).}

TCP hlavička obsahuje protichybový kontrolní součet.
Při~navazování spojení se provádí třícestný handshake:
\begin{center}
\begin{tabular}{rcl}
	klient & & server \\
	\hline
	SYN, SEQ=$X$ & $\rightarrow$ & \\
	& $\leftarrow$ & SYN, ACK=$X+1$, SEQ=$Y$ \\
	SEQ=$X+1$, ACK=$Y+1$ & $\rightarrow$ \\
\end{tabular}
\end{center}
Při~uzavírání spojení obě strany zasílají FIN, ACK.
Zamítavá odpověď je RST.


\subsection{Útoky na~TCP}

\textbf{Spoofing.}
Asociační stavem se myslí kombinace portu a~inicializačního sekvenčního čísla, které lze predikovat nebo hádat.
Ochranou je dobrá implementace TCP stacku (a PRNG v~něm), IPSec nebo např. TCP Auth Option (RFC~5925).

\textbf{Skenování portů.}
Skenováním lze zjistit jaké služby na~stanici běží, a~fingerprinting techniky umožňují detekovat i verze systému a služeb (\texttt{nmap}, \texttt{netstat}; \href{https://github.com/hackman/shijack}{\texttt{Shijack}}, \href{https://linux.die.net/man/1/hunt}{\texttt{Hunt}}).
Přímá ochrana neexistuje, je možné skenování ztížit nasazením edge proxy která serverům a jejich službám posílá jen žádaný provoz a zbytek je terminován u~ní.

\textbf{Únos TCP relace.}
Varianta spoofingu, jde o~zastavení klienta (DoS) a odhad sekvenčních čísel jeho komunikace se~serverem.

\textbf{SYN DoS.}
Zaslání mnoha SYN paketů bez následného potvrzení navázání spojení ACK zprávou.
Tím na~serveru zůstanou polootevřená spojení.
Ochranou jsou SYN cookies (hashování adres, portů a času; výsledek v~sekvenčním čísle), snížení časovače, navýšení délky fronty, recyklace nejstarších polootevřených spojení, filtrace nových spojení.


\vfill
\subsection{Protokol TLS}

\emph{Transport Layer Security}.
Jde o~mezivrstvu zajištující bezpečný přenos L7 protokolů nad~TCP.
Nabízí jedno- i dvoucestnou autentizaci s~využitím certifikačních autorit, obousměrnou symetricky šifrovanou konverzaci, kontrolu integrity a autentičnosti (HMAC, CMAC, AEAD).

Skládá se z~protokolů \emph{Record Layer Protocol} (dělení dat, komprimace, šifrování), \emph{Handshake Protocol} (autentizace stran, dohoda algoritmů a tajemství), \emph{ChangeCipherSpec} (přechod na~jinou šifrovací sadu), \emph{Alert} (signalizace).

Při~navazování se provádí čtyřcestný handshake:
\begin{center}
\begin{tabular}{rcl}
	klient & & server \\
	\hline
	ClientHello & $\rightarrow$ & \\
	& & ServerHello \\
	& & Certificate \\
	& $\leftarrow$ & ServerHelloDone \\
	ClientKeyExchange & & \\
	ChangeCipherSpec & & \\
	Finished & $\rightarrow$ & \\
	& & ChangeCipherSpec \\
	& $\leftarrow$ & Finished \\
\end{tabular}
\end{center}
Výsledná relace může obsahovat jedno nebo více TCP spojení.


\subsection{Útoky na~TLS}

\textbf{Padding Oracle} (2002).
Chyba v~návrhu SSL (MAC-then-encrypt v~CBC).
Server vyzrazuje jestli je padding správný.

\textbf{POODLE} (Padding Oracle on Downgraded Legacy Encryption; 2014).
Degradace na~SSL~3 a~návazný Padding Oracle.
Ochranou je využít autentizované režimy šifer (AES-GCM), použití encrypt-then-MAC.

\textbf{Renegotiation} (2009).
Nový handshake v~rámci existujícího spojení který nahradil ten původní.
Ochranou je ho zakázat.

\textbf{CRIME} (Compression Ratio Infoleak Made Easy; 2002).
Získání cookies v~HTTPS a SPDY spojení; kompresní metoda DEFLATE nahrazuje opakované bajty a sledováním velikostí odpovědí lze získat informace o~šifrovaném obsahu.
Ochranou je zakázat kompresi.

\textbf{BREACH} (Browser Reconnaissance and Exfiltration via Adaptive Compression of Hypertext; 2013).
Obdoba CRIME využívající HTTP kompresi místo TLS komprese.

\textbf{HeartBleed} (2014).
Chyba v~implementaci OpenSSL, konkrétně v~heartbeat zprávě.
Klient zasílá dotaz obsahující data a~jejich délku a server odpovídá stejnou zprávou.
Zranitelnost HeartBleed spočívala ve~faktu že server použil zadanou délku zprávy a mohl odpovědět částí dat ze~své paměti.


\clearpage
\section{Bezpečnost UDP (útoky, protiopatření, zabezpečení nad~protokolem UDP), bezpečnost DNS, protokol DNSSEC.}

UDP je bezstavový protokol bez~garance doručení a kontroly pořadí; hlavička obsahuje pouze porty, délku a kontrolní součet.
DTLS umožňuje data šifrovat, autentizovat i~kontrolovat.
QUIC integruje TLS~1.3 a podporuje až 0RTT komunikaci.

Záplavové a amplifikační DDoS útoky:
DNS flood, SSDP flood, NTP flood, SNMP flood, UDP fragmentation flood, VoIP flood, \dots
Ochranou je limitace UDP odpovědí a zahazování nevyžádaného UDP provozu ve~firewallu.

\textbf{Memcrashed} (2018).
\href{https://memcached.org}{Memcached} je cachovací systém urychlující načítání webových stránek (klíč--hodnota, na~způsob Redis).
Šlo o~chybu implementace umožňující až 51\,000$\times$ amplifikaci (DNS amplifikace je zhruba padesátinásobná, NTP šedesátinásobná).


\subsection{Zabezpečení nad~UDP}

UDP lze balit do~TLS.

IETF QUIC je protokol postavený nad~UDP.
Vyžaduje použití TLS~1.3, součástí handshake je také TLS handshake, což umožňuje 2RTT až 1RTT komunikaci (protože součástí třetí a čtvrté zprávy již jsou i~data).
Umožňuje dynamicky měnit adresy na~síťové a transportní vrstvě.



\subsection{Bezpečnost DNS}

DNS je hierarchický systém překladu doménových jmen a IP adres, používající port 53.
Existuje třináct kořenových serverů s~vysokou globální redundancí.

Ani v~současné době není ve~výchozím stavu šifrovaný kvůli důrazu na~rychlost; dochází k~únikům detailů o~provozu a ohrožení soukromí.
Jde o~prostor pro~tunelování provozu (SSH over DNS) i útoků.

Dosáhnout lepšího soukromí lze dosáhnout různými způsoby.
Omezení rekurze (tzv. \emph{bailiwick}) záleží na~rekurzivním resolveru a ne klientech: DNS dotaz obsahuje pouze data nutná k~vyřešení dotazu (root DNS $\rightarrow$ \texttt{cz.}, DNS pro~\texttt{.cz} $\rightarrow$ \texttt{vut.cz.}, DNS pro~\texttt{vut.cz} $\rightarrow$ \texttt{fekt.vut.cz.}, \dots).

DNS dotazy lze balit do~TLS nebo HTTPS paketů  (DoT, DoH), což prvkům infrastruktury znemožní sledovat DNS provoz a chování koncových stanic.
Dochází tím ale k~nárůstu objemu dat a~času nutnému k~vyřízení požadavku.

\textbf{DNS spoofing} je zaslání falešné odpovědi klientovi.
\textbf{DNS Cache Poisoning} je otrava cache a potenciální útok na~více klientů najednou.

\textbf{Zone transfer} je operace synchronizace Master a Slave DNS serverů v~rámci organizace.
Pokud je přenos zóny nakonfigurován nedostatečně, útočník může o~zónu požádat a získat tím seznam existujících subdomén.


\subsubsection{DNSSec}

Ochrana proti podvržení a manipulaci s~DNS dotazy.
Klient (nebo rekurzivní server) ověřuje odpovědi (původ a integritu) pomocí digitálních podpisů; nejde o~zajištění ochrany proti odposlechu nebo odepření přístupu.
Časová razítka zajišťují ochranu proti opakování.

Veřejný klíč zóny je zapsán v~doménových informacích u~nadřazené autority, čímž dochází ke~zformování řetězce důvěry.

RRSIG je podpis odpovědi, DNSKEY je veřejný klíč použitý k~podpisu, DS slouží k~ověření u~nadřazené autority, NSEC a NSEC3 prokazují neexistenci požadované domény.


\clearpage
\section{Zabezpečení v~sítích typu Low-Power Wide Area Network (kryptografické ochrany, problémy a~omezení), zabezpečení v mobilních sítích 2G -- 5G (základní principy bezpečnosti pro~2G, 3G a 4G).}



\clearpage
\section{Ochrana~soukromí v~ICT (pojmy), kryptografické metody zajištující ochranu soukromí, síťové metody poskytující ochranu soukromí a anonymizační nástroje a systémy.}

V~kontextu ICT se ochranou myslí zejména ochrana soukromí osob proti nekontrolovanému sběru, ukládání nebo uvolňování osobních údajů.

Pojmy: anonymizace (skrytí identity), pseudonymizace (částečné skrytí identity), anonymita (stav znemožňující identifikaci), osobní údaje (jméno, pohlaví, věk, datum narození, IP adresa, fotografie), citlivé osobní údaje (orientace, rodné číslo, zdravotní stav), další osobní informace (bydliště a lokalizace, finanční situace), profilování (propojování jednotlivých informací), \emph{Privacy by Default}, \emph{Privacy by Design}.

Hlavními vlastnostmi soukromí jsou \textbf{nespojitelnost}, \textbf{průhlednost} a \textbf{možnost intervence}.
Ochrana soukromí lze dělit do osmi kategorií:
\textbf{informování} uživatele o~nakládání a zpracování jejich informací, \textbf{kontrola} nad~informacemi (odstranění, (ne)souhlas), \textbf{minimalizace} a limitace sběru a zpracování, \textbf{oddělení} informací a prevence proti korelacím, \textbf{skrývání} informací (proti nepovolaným stranám), \textbf{zabstraktnění} informací (limitování detailů), \textbf{demonstrace} opatření, \textbf{vyžadování} ochrany.

Právně je ochrana soukromí zajištěna GDPR (příp. CCPA v~USA/Kalifornii),%
\footnote{Více ve~státnicových otázkách z~MPC-ODP.}
ISO/IEC 29100.


\subsection{Kryptografické metody ochrany soukromí}

E2E šifrování.
Atributová autentizace, ZK schémata, sdílená tajemství.
Skupinové, kruhové, slepé podpisy.
Homomorfní funkce a operace nad~zašifrovaným obsahem.
\emph{Multiparty computation}.
K-anonymita, maskování dat, mikrodatová ochrana.


\subsection{Síťové metody ochrany soukromí}

Privátní spojení a překlad adres: TLS, DoT/DoH.
VPN, Proxy.
Tor.


\subsection{Anonymizační nástroje a systémy}

TODO ještě něco víc?




\clearpage
\section{Forenzní analýza (hlavní cíle, základní principy, vysvětlete časové značky a časovou osu událostí).}



\clearpage
\section{Analýza škodlivého kódu (základní dělení analýz, cíle a metody statické a dynamické analýzy).}



\clearpage
\section{Penetrační testování (rozdíly v penetračním testování realizovaným metodou Ad-hoc a pomocí metodologie, hlavní cíle ASVS metodologie, bezpečnostní úrovně).}



\clearpage
\section{Testování bezpečnosti webové aplikace (vysvětlete zranitelnost Path traversal, co obsahuje testování vstupu pro nahrání souboru).}



