\section{Bezpečnost na~vrstvě L1 (bezpečnostní opatření, klíčování, dohled nad~sítí) a bezpečnost na~vrstvě L2 (bezpečnostní opatření, příklady útoků, MACsec).}



\clearpage
\section{Bezpečnost na~vrstvě L3 (bezpečnostní opatření, příklady útoků, IPsec, bezpečnost IPv6).}



\clearpage
\section{Bezpečnost TCP (útoky, protiopatření), protokol TLS -- Transport Layer Security (princip, součásti, příklady útoků).}



\clearpage
\section{Bezpečnost UDP (útoky, protiopatření, zabezpečení nad~protokolem UDP), bezpečnost DNS, protokol DNSSEC.}



\clearpage
\section{Zabezpečení v~sítích typu Low-Power Wide Area Network (kryptografické ochrany, problémy a~omezení), zabezpečení v mobilních sítích 2G -- 5G (základní principy bezpečnosti pro~2G, 3G a 4G).}



\clearpage
\section{Ochrana~soukromí v ICT (pojmy, typy anonymizace), kryptografické metody zajištující ochranu soukromí, síťové metody poskytující ochranu soukromí a anonymizační nástroje a systémy.}



\clearpage
\section{Forenzní analýza (hlavní cíle, základní principy, vysvětlete časové značky a časovou osu událostí).}



\clearpage
\section{Analýza škodlivého kódu (základní dělení analýz, cíle a metody statické a dynamické analýzy).}



\clearpage
\section{Penetrační testování (rozdíly v penetračním testování realizovaným metodou Ad-hoc a pomocí metodologie, hlavní cíle ASVS metodologie, bezpečnostní úrovně).}



\clearpage
\section{Testování bezpečnosti webové aplikace (vysvětlete zranitelnost Path traversal, co obsahuje testování vstupu pro nahrání souboru).}



