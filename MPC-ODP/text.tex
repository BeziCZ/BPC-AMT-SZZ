\section{Režim omezení odpovědnosti poskytovatelů služeb informační společnosti typu mere conduit, caching a hosting}

{}Relevantní právní předpisy:
\\\href{https://eur-lex.europa.eu/legal-content/CS/ALL/?uri=CELEX:32000L0031
}{Směrnice 2000/31/ES o~některých právních aspektech služeb informační společnosti [\dots]}
\\\href{https://www.zakonyprolidi.cz/cs/2004-480}{Zákon 480/2004 Sb. o~některých službách informační společnosti}

Poskytovatel služeb informační společnosti (ISP) se může nacházet v~jednom ze~tří režimů odpovědnosti v~závislosti na~službách které nabízí: prostý přenos, ukládání v~mezipaměti, shromažďování informací.

Tyto principy vychází především z~faktu, že ISP nemohou za~obsah odpovídat plně (protože by to pro~ně nebylo ekonomicky udržitelné na~pojištění) ani vůbec (protože by stát neměl možnost své zákony vymáhat).

Kdy služba spoluzodpovídá je stanoveno v~autorském zákoně, v~občanském zákoníku a~dalších relevantních zdrojích; Směrnice upravuje kdy poskytovatelé odpovědní nejsou.

\subsection{Mere Conduit (Prostý přenos)}

Poskytovatel není odpovědný, pokud
(a) není původcem přenosu,
(b) nevolí příjemce přenášené informace a
(c) nevolí a~nezmění obsah přenášené informace.

Jednotlivé členské státy mohou přikázat konkrétní komunikaci přerušit (v~ČR jde například o~nelicencované hazardní hry online).

\subsection{Caching, mirroring (Ukládání v~mezipaměti)}

Poskytovatel není odpovědný, pokud
(a) informaci nezměnil,
(b) vyhovuje podmínkám přístupu k~informaci,
(c) dodržuje pravidla o~aktualizaci informace,
(d) nepřekročí povolené používání technologie obecně uznávané a~používané v~průmyslu s~cílem získat údaje o~užívání a
(e) ihned přijme opatření vedoucí k~odstranění jím uložené informace.

\subsection{Hosting (Shromažďování informací)}

Hosting je služba pracující s~daty generovanými uživateli.

Poskytovatel \emph{je} odpovědný, pokud se seznámí s~protiprávním jednáním a~přesto nekoná: jde o~princip \emph{Notice -- Takedown}.
Musí konat okamžitě (\emph{expediciously}) jakmile se o~problematickém stavu dozví%
\footnote{
	V~Německu je \emph{okamžitě} chápáno z~hlediska poskytovatele.
	U~nás se takový případ ještě k~soudu nedostal.
}%
.
V~Itálii a~Španělsku musí protiprávní skutečnost oznámit stát, ve~většině členských států to může udělat kdokoliv.
V~USA to musí být ten jehož práva jsou dotčena.

% TODO Zde je možné zmínit některé příklady:
% Napster ("Ekonomický model by bez autorskoprávně chráněného obsahu nefungoval")
% Delfi AS v. Estonia ("Posuzuje se doba od protiprávního jednání, ne od jeho nahlášení, protože provozovatel musel o urážlivých komentářích vědět.")
% eBay, Uber & Airbnb a jejich umístění na škále zprostředkovatel--poskytovatel

\clearpage
\section{Aktivní povinnosti poskytovatelů služeb informační společnosti (monitoring, filtrování)}

\clearpage
\section{Pojem osobního údaje, titul ke zpracování osobních údajů, zvláštní kategorie osobních údajů}

{}Relevantní právní předpisy:
\\\href{https://eur-lex.europa.eu/legal-content/CS/ALL/?uri=CELEX:32016R0679
}{Nařízení 2016/679 o~ochraně fyzických osob v~souvislosti se zpracováním osobních údajů a o volném pohybu těchto údajů [\dots]}
\\\href{https://www.zakonyprolidi.cz/cs/2019-110}{Zákon 110/2019 Sb. o~zpracování osobních údajů}

Osobními údaji jsou veškeré informace o~identifikované/identifikovatelné fyzické osobě, kterou lze přímo či~nepřímo identifikovat.

Jde například o~jméno, identifikační číslo, lokační údaje, síťový identifikátor nebo na jeden či více zvláštních prvků fyzické, fyziologické, genetické, psychické, ekonomické, kulturní nebo společenské identity této fyzické osoby.

\subsection{Zpracování osobních údajů}

Článek 6 vymezuje zákonné tituly -- podmínky, ze~kterých alespoň jedna musí být platná, aby se jednalo o~zpracovnání dle~GDPR:

\begin{enumerate}[label=\alph*)]
\item subjekt udělil souhlas se~zpracováním,
\item zpracování je nezbytné pro~plnění smlouvy,
\item zpracování je nezbytné pro~plnění právní povinnosti správce,
\item zpracování je nezbytné pro~ochranu životně důležitých zájmů subjektu,
\item zpracování je nezbytné pro~splnění úkolu ve~veřejném zájmu nebo výkonu veřejné moci,
\item zpracování je nezbytné pro~účely oprávněných zájmů správce.
\end{enumerate}

Osobní údaje musí být zpracovávány korektně, zákonně a~transparentně.
Musí být shromažďovány pro~určité, výslovně vyjádřené legitimní účely a~nesmí být zpracovávány způsobem, který je s~těmito účely neslučitelný.
Musí být přiměřené, relevantní a~omezené na~nezbytný rozsah ve~vztahu k~účelu pro~který jsou zpracovávány (tzv. \emph{minimalizace údajů}).
Nesmí být zpracovávány po~dobu delší než nezbytnou pro~účely, pro~které jsou zpracovávány.

Zpracování pro~osobní potřebu se pod~Nařízení nevztahuje (viz důvod č.~18).
Pod~tuto kategorii spadají například čísla kontaktů v~telefonu nebo podpisy na~výsledcích tvorby umělecké činnosti%
\footnote{V~době zavádění GDPR se šířily poplašné zprávy, že není možné zveřejňovat autory výkresů v~mateřských školách.}.

\subsection{Zvláštní kategorie osobních údajů}

Zvláštní kategorie osobních údajů by dle důvodu č.~53 měly být zpracovávány pouze

\begin{itemize}
\item pro~zdravotní účely, je-li jich třeba k~dosáhnutí prospěchu fyzických osob nebo společnosti jako celku,
\item pro~účely monitorování a~varování nebo pro~účely archivace ve~veřejném zájmu,
\item pro~účely vědeckého či~historickéhov výzkumu,
\item pro~statistické účely na~základě práva Unie nebo členského státu,
\item pro~studie prováděné ve~veřejném zájmu v~oblasti veřejného zdraví.
\end{itemize}

Zakazuje se zpracování osobních údajů, které vypovídají o~rasovém či etnickém původu, politických názorech, náboženském vyznání či filozofickém přesvědčení [\dots], a~zpracování genetických údajů, biometrických údajů za~účelem jedinečné identifikace fyzické osoby a~údajů o~zdravotním stavu či o~sexuálním životě nebo sexuální orientaci fyzické osoby.

Tento zákaz neplatí, pokud se uplatní jeden z~následujících případů (viz článek 9):

\begin{enumerate}[label=\alph*)]
\item subjekt údajů udělil výslovný souhlas,
\item zpracování je nezbytné pro~účely plnění povinnosti a~výkon zvláštních práv správce [\dots] v~oblasti pracovního práva nebo práva v~oblasti sociálního zabezpečení [\dots],
\item zpracování je nutné pro~ochranu životně důležitých zájmů subjektu [\dots],
\item zpracování provádí [\dots] nadace, sdružení [\dots] pro~vnitřní účely [\dots],
\item zpracování se týká osobních údajů zjevně zveřejněných subjektem,
\item zpracování je nezbytné pro~určení, výkon nebo obhajobu právních nároků [\dots],
\item zpracování je nezbytné z~důvodu významného veřejného zájmu [\dots],
\item zpracování je nezbytné pro~účely preventivního nebo pracovního lékařství [\dots],
\item zpracování je nezbytné z důvodů veřejného zájmu v oblasti veřejného zdraví [\dots],
\item zpracování je nezbytné pro účely archivace ve veřejném zájmu, pro účely vědeckého či historického výzkumu nebo pro statistické účely [\dots].
\end{enumerate}

\clearpage
\section{Právní postavení správce a zpracovatele osobních údajů}

\clearpage
\section{Práva subjektů osobních údajů}

\clearpage
\section{Povinné subjekty dle zákona o kybernetické bezpečnosti}

\clearpage
\section{Bezpečnostní opatření, varování, reaktivní opatření a ochranná opatření dle zákona o kybernetické bezpečnosti}

\clearpage
\section{Procesní nástroje pro zajištování elektronických důkazů}

\clearpage
\section{Typy a znaky skutkových podstat počítačových trestných činů}

\clearpage
\section{Subjektivní a objektivní odpovědnost}
\subsection*{Subjektivní odpovědnost}
\begin{itemize}
    \item vzniká v případě zaviněného porušení právní odpovědnosti
    \item zavinění - klíčový pojem
    \item se subj. odpovědností spojována exkulpace = vyvinění se $\longrightarrow$ možnost prokázat, že subjekt nezavinil vznik újmy přestože jsou splněny dané předpoklady
    \item musí se prokázat:
    \begin{itemize}
        \item subjekt - kdo a zda může být zodpovědný, pachatel
        \item objekt - právem chráněný zájem/právo (i virtuální věci)
        \item subjektivní stránka - zavinění, buď úmysl nebo nedbalost
        \item objektivní stránka - škodlivý následek, protiprávní následek, kauzální nexus (příčinná souvislost)
    \end{itemize}
\end{itemize}

\subsection*{Objektivní odpovědnost}
\begin{itemize}
    \item odpovědnost za protiprávní stav, událost
    \item nejde o odpovědnost za protiprávní jednání, ale za škodlivý následek
    \item možnost liberace = možnost zprostit se objektivní odpovědnosti; vyšší moc $\longrightarrow$ ke škodě by došlo i bez přičinění strany 
    \item pokud právem nelze  užít liberaci $\longrightarrow$ absolutní objektivní odpovědnost
    \item např. odpovědnost za vady, která vzniká nezávisle na zavinění zhotovitelem, příčina vady musí spočívat v něčem jiném, než jsou okolnosti způsobené objednatelem
    \item  u trestní odpovědnosti není objektivní odpovědnost - vždy se dokazuje zavinění 
\end{itemize} 

\subsection*{Prvky právní odpovědnosti}
\begin{itemize}
    \item = subjektivní stránka subjektivní odpovědnosti \\
\end{itemize}

\underline{Úmysl}
\begin{itemize}
    \item přímý
    \begin{itemize}
        \item ví, že se dopouští protipr. jednání, chce způsobit následek
        \item např. vražda
    \end{itemize}
    \item nepřímý
    \begin{itemize}
        \item ví, že se dopouští protipr. jednání, ale nemá v úmyslu způsobit škodlivý následek
        \item je možnost, že ví, že následek může nastat
        \item např. chtěl zranit, ale zabil \\
    \end{itemize}
\end{itemize}

\underline{Nedbalost}
\begin{itemize}
    \item vědomá
    \begin{itemize}
        \item ví, že se může něco stát a spoléhá na to, že se nic nestane
    \end{itemize}
    \item nevědomá
    \begin{itemize}
        \item neví, že porušuje právní normu a dopouští s protipr. jednání
        \item mohl a měl vědět, že se něco může stát
    \end{itemize}
\end{itemize}

\subsection*{Právní skutečnosti}
\begin{itemize}
    \item = okolnosti, se kterými právní norma spojuje vznik, změnu nebo zánik právního vztahu \\
\end{itemize}

\underline{Prokazatelné}
\begin{itemize}
    \item nutné prokázat, některé se neprokazují snadno
    \item lze prokázat i záporem (např. smrt člověka)
    \item korespondenční teorie pravdivosti - každou ze skutečností je třeba prokázat
    \item verifikace = prokázání důkazem X falzifikace = prokázání antiteze
    \item koherenční teorie - nemáme přímý ani falzifikovatelný důkaz, ale existují nepřímé důkazy (otisk prstů, svědectví)
    \item alibi může způsobit nekoherenci
    \item nepřímé důkazy - mají vztah s prokazovanou skutečností \\
\end{itemize}

\underline{Konsenzuální teorie}
\begin{itemize}
    \item skutečnost, že společnost se chová podle jistých nepsaných pravidel \\
\end{itemize}

\underline{Předpokládané} 
\begin{itemize}
    \item prokazuji předpoklad, skutečnost konstruuje právní norma
    \item domněnky - otcem je manžel matky, ale nemusí být
    \item fikce - skutečnost nenastala, ale právo říká, že nastala (např. fikce doručení - 10. den po oznámení) \\
\end{itemize}

\underline{Známé}
\begin{itemize}
    \item notoriety - všeobecně známé
    \item známé z úřední povinnosti (obchodní rejstřík)
    \item známé z rozhodovací činnosti - známé z praxe (sazba ČNB)
\end{itemize}
\clearpage


