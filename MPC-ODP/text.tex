\section{Režim omezení odpovědnosti poskytovatelů služeb informační společnosti typu mere conduit, caching a hosting}

{}Relevantní právní předpisy:
\\\href{https://eur-lex.europa.eu/legal-content/CS/ALL/?uri=CELEX:32000L0031
}{Směrnice 2000/31/ES o~některých právních aspektech služeb informační společnosti [\dots]}
\\\href{https://www.zakonyprolidi.cz/cs/2004-480}{Zákon 480/2004 Sb. o~některých službách informační společnosti}

Poskytovatel služeb informační společnosti (ISP) se může nacházet v~jednom ze~tří režimů odpovědnosti v~závislosti na~službách které nabízí: prostý přenos, ukládání v~mezipaměti, shromažďování informací.

Tyto principy vychází především z~faktu, že ISP nemohou za~obsah odpovídat plně (protože by to pro~ně nebylo ekonomicky udržitelné na~pojištění) ani vůbec (protože by stát neměl možnost své zákony vymáhat).

Kdy služba spoluzodpovídá je stanoveno v~autorském zákoně, v~občanském zákoníku a~dalších relevantních zdrojích; Směrnice upravuje kdy poskytovatelé odpovědní nejsou.

\subsection{Mere Conduit (Prostý přenos)}

Poskytovatel není odpovědný, pokud
(a) není původcem přenosu,
(b) nevolí příjemce přenášené informace a
(c) nevolí a~nezmění obsah přenášené informace.

Jednotlivé členské státy mohou přikázat konkrétní komunikaci přerušit (v~ČR jde například o~nelicencované hazardní hry online).

\subsection{Caching, mirroring (Ukládání v~mezipaměti)}

Poskytovatel není odpovědný, pokud
(a) informaci nezměnil,
(b) vyhovuje podmínkám přístupu k~informaci,
(c) dodržuje pravidla o~aktualizaci informace,
(d) nepřekročí povolené používání technologie obecně uznávané a~používané v~průmyslu s~cílem získat údaje o~užívání a
(e) ihned přijme opatření vedoucí k~odstranění jím uložené informace.

\subsection{Hosting (Shromažďování informací)}

Hosting je služba pracující s~daty generovanými uživateli.

Poskytovatel \emph{je} odpovědný, pokud se seznámí s~protiprávním jednáním a~přesto nekoná: jde o~princip \emph{Notice -- Takedown}.
Musí konat okamžitě (\emph{expediciously}) jakmile se o~problematickém stavu dozví%
\footnote{
	V~Německu je \emph{okamžitě} chápáno z~hlediska poskytovatele.
	U~nás se takový případ ještě k~soudu nedostal.
}%
.
V~Itálii a~Španělsku musí protiprávní skutečnost oznámit stát, ve~většině členských států to může udělat kdokoliv.
V~USA to musí být ten jehož práva jsou dotčena.

% TODO Zde je možné zmínit některé příklady:
% Napster ("Ekonomický model by bez autorskoprávně chráněného obsahu nefungoval")
% Delfi AS v. Estonia ("Posuzuje se doba od protiprávního jednání, ne od jeho nahlášení, protože provozovatel musel o urážlivých komentářích vědět.")
% eBay, Uber & Airbnb a jejich umístění na škále zprostředkovatel--poskytovatel

\clearpage
\section{Aktivní povinnosti poskytovatelů služeb informační společnosti (monitoring, filtrování)}

\clearpage
\section{Pojem osobního údaje, titul ke zpracování osobních údajů, zvláštní kategorie osobních údajů}

\clearpage
\section{Právní postavení správce a zpracovatele osobních údajů}

\clearpage
\section{Práva subjektů osobních údajů}

\clearpage
\section{Povinné subjekty dle zákona o kybernetické bezpečnosti}

\clearpage
\section{Bezpečnostní opatření, varování, reaktivní opatření a ochranná opatření dle zákona o kybernetické bezpečnosti}

\clearpage
\section{Procesní nástroje pro zajištování elektronických důkazů}

\clearpage
\section{Typy a znaky skutkových podstat počítačových trestných činů}

\clearpage
\section{Subjektivní a objektivní odpovědnost}
