\section{Základní nelinearity a popis nelineárních systémů (popis statických a dynamických nelineárních systémů,
vliv parazitních nelinearit na průběh regulačního děje)}


\subsection{bez dynamiky}
Systém reaguje na odezvu okamžitě, nebo je jeho časová konstanta výrazně nižší něž zbylé čas. konstanty.
Dělí se na :

\begin{itemize}
    \item bez paměti
    \item s pamětí
\end{itemize}

\subsubsection*{nelinearita bez paměti}

Dá se vyjádřit jako funkce 
\begin{equation}
    y=f(u)
\end{equation}

Na základě aktuální hodnoty se určí výstup. Dá se jednoduše linearizovat rozvojem da Taylorovi řady, nebo do polynomu.
popsána funkcí, grafem, tabulkou.
\\
např. :  saturace, tření\dots
\\
\\
\subsubsection*{nelinearita z pamětí}
Výstup záleží i na předchozích stavech, nelze jednoduše popsat jako funkce, často se popisuje slovně nebo jako algoritmus.
\\
např.: vůle v převodech, relé s hysterezí

\subsection*{Základní nelinearity}

\subsubsection*{nasycení(saturace)}
\begin{itemize}
    \item nejběžnější nelinearita (obsahuje ji v podstatě každý reálný systém)
    \item dá se s ní pracovat jako s po částech lineární funkcí
    \item pokud zajistíme že se systém nedostane z <b,a> lze prát jako lin. f-ce 
    \item např.: nádrž co přeteče
    \item (je jedním z faktorů pro vznik wind up jevu)
    \item bez paměti
\end{itemize}
\includegraphics[]{img/saturace.png}

\subsubsection*{necitlivost}
\begin{itemize}
    \item nejčasttěji u mech systémů (projev tření a mech. nepřesností)
    \item např.: DC motor se roztočí až od určitého napětí
    \item občas se vkládá do reg. umělěle aby se omezili oscilace
    \item po částech lineární (většinou)
    \item bez paměti
\end{itemize}
\includegraphics{img/necitlivost.png}

\subsubsection*{vůle v převodech (hystereze)}
\begin{itemize}
    \item při změně pohybu chvíli trvá než se něco začne dít
    \item v převodovkách, nebo při magnetizaci železa (hysterezní smyčka- v otázce MVE mag. měření)
    \item v převodovce lze odstranit tak že druhý motor tlačí vždycky proti
    \item rádi se na to ptají
    \item většinou nelze linearizovat
    \item s pamětí
\end{itemize}
\includegraphics{img/vule.png}

\subsubsection*{tření}
\begin{itemize}
    \item v mech systémech
    \item špatně se linearizuje v okolí 0
    \item bez paměti
\end{itemize}
\includegraphics{img/treni.png}

\subsection*{releové charakteistiky}
\begin{itemize}
    \item s pamětí i bez
    \item požití jako regulátor
\end{itemize}
\includegraphics*{img/rele.png}

\subsection{s dynamikou}
popsány stavovými rovnicemi:
\[
    \frac{dx}{dt}=f(x,u)\]
    \[y=g(x,u)\]

\begin{itemize}   
    \item časově variantní
    \item časově invariantí (parametry se v čase nemění)
\end{itemize}
%----------------------------------
\newpage
\section{Ustálené chování nelineárních dynamických systémů (rovnovážné stavy, mezní cyklus, metoda
harmonické rovnováhy, stabilita mezního cyklu)}

Není v otázce přímo zmíněný ale asi dobrý vědět.

\includegraphics[scale = 0.8]{img/trajektorie.png}
{
\it stavová trajektorie (po systém 2. řádu) je 3d graf ukazuje jak se v čase mění vnitřní stavy systému.
Požívá se průmět stavové trajektorie, ukazuje jak se bude systém chovat, šipkou naznačeno odkud kam se bude v čase pohybovat.
}
\subsection{rovnovážný stav}
Stav systému, ktrý se v čase nemění. x=0 Průmět stavové trajektorie je bod. Určíme
ho jako \[ \dot{x} = 0 \]
Systém může mít o vice rovnovážných stavů, také nemusí mít žádný.

Rovnovážné stavy mohu být:
\begin{itemize}
    \item izolované - pokud kolem něj existuje jeho malé okolí ve kterém se nenachází další rovnovážný stav.
    \item neizolované
\end{itemize}

U rovnovážných stavů pak můžeme řešit jejich stabilitu, pokud se systém po drobném vychýlení do rovnaného 
stavu vrátí je rovnovážný stav stabilní.
O stabilitě/nestabilitě lze rozhodnout linearizací rozvojem do Taylorovi řady (podle polohy pólů náhrady)

\subsubsection{mezní cyklus}
Mezní cyklus nastává pokud se stavy systému periodicky opakují, průmět stav trajektorie je uzavřený obrazec (kruh).
Matematicky řečeno :
\[x(t+T)=x(t)\]
t-čas \\ T-perioda\\

Mezní cyklus muže být stabilní polostabilní nebo nestabilní:  

\includegraphics{img/mez_cykly.png}

Zda vzniknou mezní cykly jde určit metodou harmonické rovnováhy
\subsubsection*{metoda harmonické rovnováhy}
Vyžívá ekvivalentního přenosu nelinearity ke zkoumání mezních cyklů.

\subsubsection*{ekvivaletní přenos}
nelinearitu nahradíme ekvivalentním přenosem $N_e$, ten je závislá na Amplitudě vstupního signálu A.

Ekvivalentní přenos získáme  tak, že odezvu nelinearity na harmonický průběh o amplitudě A, převedeme pomocí Fourierovy transformace na jeho ekvivalentní přenos. U Fourierovy transformace uvažujeme pouze 1. harmonickou, 
protože se bere že další části obvodu mají charakter dolní propusti. Matematicky řečeno:
\begin{equation*}
    N_e= \frac{a_1+j\cdot b_1}{A}
\end{equation*}
kde
\begin{equation*}
    a_1=\frac{2}{T} \int_{0}^{T} f(e) \cdot sin(\omega t) dt
\end{equation*}

\begin{equation*}
    a_1=\frac{2}{T} \int_{0}^{T} f(e) \cdot cos(\omega t) dt
\end{equation*}

{\bf Tuto náhradu lze použít pouze pokud}
\begin{itemize}
    \item je dalším prvkem v regulačním obvodu systém typu dolní propust, který dostatečně potlačuje vyšší harmonické
    \item nelinearita při vstupním signálu $e=A\cdot sin (\omega t) $ negeneruje stejnosměrnou složku
\end{itemize}


Zjištění mezních cyklů 

\includegraphics{img/harm.rovnovha.png}

Pro určení mezních cyklů musí být systém jako je na obrázku, kde f je nelinearita a G je lin přenos typu dolní propust.

Mezní cykly v systému vzniknou pokud bude "$ F_o =-1 $", tedy pokud bude výstup systému 
roven jeho výstupu posunutému o 180°, matematicky :
\begin{equation*}
    A= N_e \cdot G \cdot A
\end{equation*}
Úpravou dostaneme podmínku vzniku mezních cyklů jako :
\begin{equation*}
    G=\frac{-1}{N_e}
\end{equation*}
Tato rovnice často nelze numericky řešit a musí se řešit graficky.

\includegraphics{img/garf.reseni.png}

dál lze touto metodou určit jestli je mezní cyklus stabilní, pomocí modifikovaného Nyquistova kritéria :
 například u bodu 1, pokud se sníží A dostaneme se do bodu 1'' pokud je charakteristika G nalevo do něj systém bude nestabilní tj amplituda se opět zvýší, pokud A zvýšíme dostane se do bodu 1' systém bude stabilní a amplituda se zmenší, mezní cyklus v bodě 1 je tedy stabilní.

 Tupě řečeno mezní cyklus je stabilní pokud se po snížení A nachází bod 1'' napravo od G a při zvýšení A se bod 1' nachází nalevo do G.
 mezní cyklus v bodě 1 je stabilní mezní cyklus v bodě 2 není.
 
 \includegraphics{img/stab.mez.cyklu.png}

 \section{Stabilita nelineárních systémů (definice, metody vyšetření, věty o nestabilitě, stabilita uzavřené regulační
 smyčky)}

 stabilitu určujeme podle Ljapunovových kritérií, ty platí pouze pro neřízené systémy.
 podle Ljapunova existují následující stability :

 \subsection{lokálně stabilní}
 Pokud v (libovolně malém) okolí $\varepsilon $ rovnovážného stavu $x_0$ vybereme počáteční bod $\delta $ jeho trajektorie dané okolí nepustí.

 \includegraphics{img/lok.stabilní.png}

 podle Václavka :

 \includegraphics{img/vaclave.lok.png}

 \subsection{lokálně asymptoticky stabilní}
Systém je lokálně stabilní viz předchozí odstavec a zároveň platí že pro $t=\infty$ se systém dostane do $x_0$

\includegraphics[scale = 0.6]{img/lok.asypt.stab.png}

podle Václavka:

\includegraphics{img/vaclavek. lok.asypt..png}

\subsection{globálně  asymptoticky stabilní}
Pokud je systém lokálně asymptoticky stabilní a jeho okolí přitažlivosti (pokud v daném daného okolí umístíme systém skončí v $x_0$) zabírá 
celý stavový prostor.

jinými slovy pro libovolný bod stavového prostoru systém skončí v $x_0$.

podle Václavka

\includegraphics{img/vaclavek.golbal.png}

Tato podmínka je docela tvrdá, proto se zavádí Praktická stabilita, systém je prakticky stabilní v oblasti $\Omega$ pokud je lokálně asymptoticky stabilní a má oblast přitažlivosti $\Omega$.


\subsection*{určení stability}
Jedná pouze o podminky postačující pro satbilitu, pokud tedy systém podmínce nevyhoví nelze o něm říct že je nestabilní.
\subsubsection{I Ljapunovovo kritériu stability}
Určuje stabilitu systému na zákldadě stbility linearizovéhé náhardy v okolí rovnovážnéh o stavu.

\noindent \includegraphics[scale = 0.6]{img/Ljap.1.png}


\subsubsection{II Ljapunovovo kritériu stability}
Určíme si fukci V aby byla postivně definitní (PD), spočítáme W jako gradiet V (vpodstatě spíš derivaci ale ve skriptech to mají nazváno gradiet). Podle toje jestli je w ND , NSD urřime stabilitu.

\includegraphics[scale = 0.6]{img/Ljap.2.png}

V můžeme volit náhodně nebo lze využít {\bf Krasovského metodou}, kde V zvolíme jako součin námi zvolené matice L a funkcí f, ktré ziskáme ze stavových rovic.
L musí být symetrická a PD. Matamatickými čáry pak dostaneme že W bude ND pokud T bude PD.
T spočítáme jako :
\begin{equation*}
    T=J^TL+JL
\end{equation*}

další metodu volby V je {\bf metoda variabilního gardietu } to ale doufám že umět nemusíme.

\subsubsection{Popovo kritérium stability}
Aby byl podle popova systém stabilní musí být splneno:
\begin{itemize}
    \item systém je neřízený
    \item systém  je v kofigraci jak ne an obr.

        \includegraphics{img/popov.schem.png}
    \item hodnota nelinerity v 0 je 0 $f(0)=0$
    \item průbeh nellinearity je "pod" přímkou procházejíci počátkem se směrnicí k

        \includegraphics{img/popov_nelilin.png}
    \item modnifikovaná frekvenční charakteristika $f^*=Re{F}+j\omega Im{F}$ (F je frekvenční charakteristika zkoumaného systému) se musí nacháze pod popovovou přímkou,
    ta může mít libovolný sklon a musí procházet bodem -1/k

        \includegraphics{img/popov_frek.png}
\end{itemize}
pod jsou všechny podminky splněny systém je globálně asymptoticky stabilní.


\subsection{věty o nestabilitě}
Opět se jedná pouze o posatčující podmínky, pokud jimsystém nevyhoví nevíme nic.

\subsubsection{I Ljapunovova věta o o nestbilitě}
podud exzistuje funkce V pro ktrou platí :
\begin{itemize}
    \item V(0)=0
    \item $\frac{dV}{dt}$ je PD
    \item V není ND nebo NSD v libovolně málem okolí počátku
\end{itemize}
poto je rovnovážný bod v počátku nestabilní.


\subsubsection{II Ljapunovova věta o o nestbilitě}
pokud V není ND , nebo NSD, a její drivace je PSD, je rovnovážný bod v počátku nestabilní.
* V musí být spojitá a musí mít spojitou 1. derivaci.
\subsubsection{četajevoa věta o nestabilitě}
vůbec nechápu

%------------------------
\section{Reléová regulace (on-off regulátory, řízení v klouzavém režimu).
}

Jako regulátor je použito relé, nebo jiný spínací prvek, chovající se stejně. 
Použije se pokud akční zásah nabývá jen 2 hodnot (on-off control),nebo pokud akční veličina max kladné nebo max záporné hodnoty (bang-bang control).

Nejtypičtější využití regulace teploty (žehlička, topení), tlak v kopresorech se zásobníkem.
Možné variaty relé josu o dvě otázky víš.

Pokud bychom chtěli regulovat např teplotu pomocí klasického relé, neustále by spínalo  (i vlivem šumu), proto je lepší použít relé s hysterezi, podle šížky hasterezre můžeme nastavit s jakou periodou bude kmitat.

Pokud nám u řízení pomocí zelíé s hysterezí nevyhovuje nastvení hystereze a z nějakýho (nepochopielnýho) duvodu ji nemůžeme měnit lze chvání upravi přidáním setrvačného článku do zpětné vazby.
Musí platit že oba přepínací body jsou větší než 0, potom při sepnutí relé dojde ke zpoždění informace ve zpětný vazbě a relé sepne zpět až pozdeji. Časová konstanta setrvačného článku musí být malá.


\includegraphics{img/rele_reg.png}

{ \it
Pokud řídím polohu a chci stihnou zastavit včas, může být přijená další zpětná vazba od rychlosti, tuoudle hodnotou si upravím kdy bude relé spínat.
}

pro návrh releového regulátoru lze využít např Ljapunova...

Pro složitější systémy se využívá {\bf řízení v klouzavém režimu}
Relé ja nastaveno tak, aby se systém dosatl na přepínací rozhraní  po přepínacím rozhraní došel až do 0.

 \includegraphics{img/kluzavy_obr.png}

 \begin{itemize}
    \item systém může obsahovat neurčitost -> jedná se o robustní řízení
    \item musím být schopen měřit aktuání hodnotu vnitřních stavů
    \item na přepínacím rozhraní kmitá relé nesmyslně ryhcle, proto se nahrazuje něčím pozvolnějším 

    \includegraphics[scale = 0.5]{img/kluzavy_rele.png}

 \end{itemize}

obeně postup návrhu řízení v klouzavém režimu
\begin{enumerate}
    \item popis systému převedeme na požadovaný tvar (odělené neurčitosti $\delta $ od zbylých závislostí)
    
    \includegraphics[scale =0.4]{img/klouzavy_tavr.png}
    \item navrhneme žízení systému na který může epůsobit přímo vstupem - tím definujeme přepínací rozhraní
    \item zavedeme proměnou co určuje odchylku od přepínacího rozharní, její derivcí získáme "systém", který musí směřovat do 0 (navrhneme pro něj takové řízení)
    \item určíme ekvivalentí žizení $u_{eq}$, tak aby v rovnici došlol k elimiaci znmých člneů
    \item dosadíme získáme $\Delta$
    \item urrčíme max $\Delta$ 
    \item dosadíme, určíme řízení (předppis pro regulátro) V

\end{enumerate}

%-----------------------------------
\section{0. Linearizace nelineárních dynamických systémů (rozvoj do Taylorovy řady, exaktní zpětnovazební
linearizace).
}