\section{Základní nelinearity a popis nelineárních systémů (popis statických a dynamických nelineárních systémů,
  vliv parazitních nelinearit na průběh regulačního děje)}

\subsection{Nelineární systémy bez dynamiky}
Systém reaguje na odezvu okamžitě bez přechodného děje a zpoždění, nebo je jeho časová konstanta výrazně nižší něž zbylé časové konstanty.
Dělí se na:
\begin{itemize}
    \item bez paměti
    \item s pamětí
\end{itemize}

\subsubsection*{Systém bez paměti}
Vztah mezi vstupem a výstupem se dá vyjádřit jako funkce:
\begin{equation}
    y=f(u)
\end{equation}
Takové systémy můžeme považovat za systémy s jedním stavem, který se němění. Stavový prostor je tedy množina o jednom prvku.\\
Na základě aktuální hodnoty se určí výstup. Dá se jednoduše linearizovat rozvojem do Taylorovy řady nebo do polynomu. Může být popsána funkcí, grafem, tabulkou.\\
Pokud je systém zadán grafem nebo tabulkou je potřeba pro výpočet provést náhradu interpolační funkcí.\\
Metody linearizace:
\begin{itemize}
    \item Metoda nejmenších čtverců
    \item Taylorova řada
    \item Systémy po částech lineární
\end{itemize}
Například: saturace, tření\dots

\subsubsection*{Systémy s pamětí}
Výstup záleží i na předchozích stavech, nelze jednoduše popsat jako funkce, často se popisuje slovně nebo jako algoritmus.
\\
Například vůle v převodech, relé s hysterezí

\subsection*{Základní nelinearity}

\subsubsection*{Nasycení(saturace)}
\begin{itemize}
    \item nejběžnější nelinearita (obsahuje ji v podstatě každý reálný systém)
    \item dá se s ní pracovat jako s po částech lineární funkcí
    \item pokud zajistíme že se systém nedostane z <b,a> lze brát jako lineární funkce
    \item např.: nádrž co přeteče
    \item (je jedním z faktorů pro vznik wind up jevu)
    \item bez paměti
\end{itemize}
\includegraphics[]{img/saturace.png}

\subsubsection*{Necitlivost}
\begin{itemize}
    \item nejčastěji u mechanických systémů (projev tření a mechanické nepřesností)
    \item např.: DC motor se roztočí až od určitého napětí
    \item občas se vkládá do regulace uměle aby se omezily oscilace
    \item po částech lineární (většinou)
    \item bez paměti
\end{itemize}
\includegraphics{img/necitlivost.png}

\subsubsection*{Vůle v převodech (hystereze)}
\begin{itemize}
    \item při změně pohybu chvíli trvá než se něco začne dít
    \item v převodovkách, nebo při magnetizaci železa (hysterezní smyčka- v otázce MVE mag. měření)
    \item v převodovce lze odstranit tak že druhý motor tlačí vždycky proti
    \item rádi se na to ptají
    \item většinou nelze linearizovat
    \item s pamětí
\end{itemize}
\includegraphics{img/vule.png}

\subsubsection*{Tření}
\begin{itemize}
    \item Představuje třecí síly a momenty, nejčastěji v mechanických systémech
    \item Popis chování se aproximuje
    \item špatně se linearizuje v okolí 0
    \item bez paměti
    \item \(y = \begin{cases}
              y_a + xtg\alpha & x > 0 \\
              y_b - xtg\beta  & x < 0
          \end{cases}\)
\end{itemize}
\includegraphics{img/treni.png}

\subsection*{Reléové charakteristiky}
\begin{itemize}
    \item s pamětí i bez
    \item požití jako regulátor
    \item Relé bez hystereze
          \begin{itemize}
              \item Ideální 2-stavové relé
              \item Nelze linearizovat v bodě x=0
          \end{itemize}
    \item Relé s hysterezí
          \begin{itemize}
              \item S pamětí
              \item Například regulátory teploty(lednička, pokojové termostaty), hystereze se využívá k omezení četnosti přepínání.
          \end{itemize}
\end{itemize}
\includegraphics*{img/rele.png}
\subsection*{Obecná nelinearita}
Lze vyjádřit obecnou funkcí \(y = f(x)\).\\
Například charakteristiky elektronických prvků(diody, tyristory), fyzikální jevy(aerodynamická odporová síla, gravitační zákon)
\begin{figure}[h!]
    \centering
    \includegraphics[scale = 0.5]{img/ObecNelin.png}
\end{figure}
\subsection{Nelineární systémy s dynamikou}
Není možné je popsat statickými převodními charakteristikami, které vyjadřují závislost mezi okamžitou hodnotou výstupního v vstupního signálu. Musíme posuzovat jejich vývoj v čase.\\
Popsány stavovými rovnicemi:
\[\frac{dx}{dt}=\mathbf{f}(\mathbf{x},\mathbf{u})\]
\[\mathbf{y}=\mathbf{g}(\mathbf{x},\mathbf{u})\]

\begin{itemize}
    \item časově variantní - \(\frac{dx}{dt}=\mathbf{f}(\mathbf{x},\mathbf{u},t)\) a \(\mathbf{y}=\mathbf{g}(\mathbf{x},\mathbf{u},t)\)
    \item časově invariantí (parametry se v čase nemění)
\end{itemize}
Často se dá systém rozdělit na lineární a nelineární část:
\begin{figure}[h!]
    \centering
    \includegraphics[scale = 0.4]{img/RozdNelLin.png}
\end{figure}
%----------------------------------
\newpage
\section{Ustálené chování nelineárních dynamických systémů (rovnovážné stavy, mezní cyklus, metoda
  harmonické rovnováhy, stabilita mezního cyklu)}

Není v otázce přímo zmíněný ale asi dobrý vědět.

\includegraphics[scale = 0.8]{img/trajektorie.png}
{
    \it stavová trajektorie (po systém 2. řádu) je 3d graf ukazuje jak se v čase mění vnitřní stavy systému.
    Požívá se průmět stavové trajektorie, ukazuje jak se bude systém chovat, šipkou naznačeno odkud kam se bude v čase pohybovat.
}
\subsection{Rovnovážný stav}
Stav systému, který se v čase nemění. Průmět stavové trajektorie do stavového prostoru je bod, také označován jako singulární bod. Určíme
ho jako \[ \dot{x} = 0 \]
Systém může mít více rovnovážných stavů, také nemusí mít žádný.

Rovnovážné stavy mohou být:
\begin{itemize}
    \item izolované - pokud kolem něj existuje jeho malé okolí ve kterém se nenachází další rovnovážný stav.
    \item neizolované
\end{itemize}

U rovnovážných stavů pak můžeme řešit jejich stabilitu, pokud se systém po drobném vychýlení do rovnovážného stavu vrátí je rovnovážný stav stabilní. O stabilitě/nestabilitě lze rozhodnout linearizací rozvojem do Taylorovy řady (podle polohy pólů náhrady).

\subsubsection*{Stavový portrét}
Pomocí stavového portrétu si můžeme graficky znázornit čáry, pomocí kterých se dá následně určit stabilita mezního cyklu. Počítá se směrnice z n stavových rovnic(n počet rovin). V našem případě nejčastěji $k = \frac{\frac{dx_2}{dt}}{\frac{dx_1}{dt}}$ kde podle derivace zjišťujeme směrnici. Dá se určit taky podle znaménka, když si do zlomku dosadíme konkrétní bod.\\
Když se nachází ve stavové rovnici nelinearita, musíme zavést substituci a dostat rovnice do tvaru, kdy na y je \textit{a} a na ose x je jeho derivace.\\
Stavová trajektorie se používá tam, kde by byl analytický výpočet náročný, nebo nám stačí pochopit chování systému.

\subsection{Mezní cyklus}
Mezní cyklus nastává pokud se stavy systému periodicky opakují, průmět stav trajektorie je uzavřený obrazec (kruh).
Matematicky řečeno : 
\[x(t+T)=x(t)\]
t-čas \\ T-perioda\\

Mezní cyklus muže být stabilní polostabilní nebo nestabilní:
\begin{itemize}
    \item Stabilní je takový mezní cyklus, kterého trajektorie vycházející z počátečních podmínek systému v okolí izolovaného mezního cyklu směřují do vnitřku tohoto mezního cyklu.
    \item Nestabilní je takový, jehož trajektorie se od mezního cyklu vzdaluje.
    \item Polostabilní je takový mezní cyklus, kterého trajektorie se na jedné straně mezního cyklu vzdaluje a na druhhé straně do něho vchází.
\end{itemize}
\includegraphics{img/mez_cykly.png}

Zda vzniknou mezní cykly jde určit metodou harmonické rovnováhy.\\
Stejně jako u rovnovážných stavů můžene rozhodovat o jeho stabilitě a izolovanosti.\\
\subsubsection{Vyšetření existence mezních stavů}
\subsubsection*{Bendixonův teorém}
Nechť je \(\mathcal{D}\) jednoduše souvislá oblast \(\mathcal{D} \subset \mathbb{R}^2 \) taková, že divergence:
\begin{center}
    \(\nabla \mathbf{f}(\mathbf{x}) = \frac{\partial f_1}{\partial x_1}(x_1,x_2) + \frac{\partial f_2}{\partial x_2}(x_1,x_2)\)
\end{center}
pole směrových vektorů systému:
\begin{center}
    \(\frac{dx_1}{dt} = f_1(x_1,x_2)\)\\
    \(\frac{dx_2}{dt} = f_2(x_1,x_2)\)
\end{center}
není identicky rovna nule v jakékoliv podoblasti \(\mathcal{D}\) a nemění v \(\mathcal{D}\) znaménko, pak \(\mathcal{D}\) neobsahuje žádné uzavřené trajektorie.\\
Jednoduše souvislou oblast si lze představit jako deformovaný kruh. Teorém udává podmínku postačující pro neexistenci mezního cyklu.\\
\subsubsection*{Pointcaré-Bendixonův teorém}
Můžeme-li nalézt uzavřenou souvislou ohraničenou oblast \(\mathcal{M} \subset \mathbb{R}^2\) takovou, že neobsahuje žádné singulární body a takovou, že nějaká trajektorie systému v této oblasti do ní vstupuje a s rostoucím časem už v ní zůstává, potom tato oblast obsahuje alespoň jedno periodické řešení systému.\\
Postačující podmínkou pro to aby taková trajektorie existovala je, aby pole směrových vektorů směřovalo v každém bodě hranice oblasti M dovnitř této oblasti. Zobrazeno na obrázku.\\
\begin{figure}[h!]
    \centering
    \includegraphics[scale = 0.5]{img/PointBendix.png}
\end{figure}

\subsubsection*{Indexové teorémy}
Předpoklady:
\begin{itemize}
    \item \(\mathcal{D} \subset \mathbb{R}^2 \) je spojitá oblast
    \item \(mathbf{f}: \mathbb{R}^2 \rightarrow \mathbb{R}^2\) je zobrazení pole směrových vektorů
    \item \(\mathcal{D}\) obsahuje pouze izolované singulární body \(\frac{d\mathbf{x}}{dt} = \mathbf{f}(\mathbf{x})\)
\end{itemize}
Nechť je \(\mathcal{J}\) je jednoduchá spojitá pozitivně orientovaná Jordanova křivka uvnitř \(\mathcal{D}\), která neprochází žádným singulárním bodem systému. Pak indexem této Jordanovy křivky je vzhledem k poli směrových vektorů \(\mathbf{f}\) rozumíme číslo \(I_f(\mathcal{J})\) definované integrálem
\begin{center}
    \(I_f(\mathcal{J}) = \frac{1}{2\pi} \oint d\Theta_f(x_1,x_2)\)
\end{center}
Pod touto křivkou si můžeme představit deformovanou kružinici, při kladné orientaci obcházíme při integraci tuto křivku proti směru hodinových ručiček. Integrál pak udává počet celých otáček vektoru po křivce  \(\mathcal{J}\).\\
Pointcareův index:
\begin{itemize}
    \item Index křivky, která neobsahuje žádný singulární bod je 0
    \item Pokud obsahuje singulární bod typu střed, ohnisko či uzel je 1
    \item Pokud obsahuje singulární bod typu sedlo je -1
\end{itemize}
Pokud má systém jen izolované singulární body, pak každý jeho mezní cyklus obsahuje uvnitř konečný počet těchto bodů takový, že součet jejich indexů je 1.\\
\subsection{Metoda harmonické rovnováhy}
Zabezpečuje lineární náhradu systému, který se pohybuje po ustálené trajektorii(mezním cyklu).\\
Vyžívá ekvivalentního přenosu nelinearity ke zkoumání mezních cyklů.

\subsubsection*{Ekvivaletní přenos}
Nelinearitu nahradíme ekvivalentním přenosem $N_e$, ten je závislý na amplitudě vstupního signálu A.

Ekvivalentní přenos získáme  tak, že odezvu nelinearity na harmonický průběh o amplitudě A, převedeme pomocí Fourierovy transformace na jeho ekvivalentní přenos. U Fourierovy transformace uvažujeme pouze 1. harmonickou,
protože se bere že další části obvodu mají charakter dolní propusti. Matematicky řečeno:
\begin{equation*}
    N_e= \frac{a_1+j\cdot b_1}{A}
\end{equation*}
kde
\begin{equation*}
    a_1=\frac{2}{T} \int_{0}^{T} f(e) \cdot sin(\omega t) dt
\end{equation*}

\begin{equation*}
    b_1=\frac{2}{T} \int_{0}^{T} f(e) \cdot cos(\omega t) dt
\end{equation*}

{\bf Tuto náhradu lze použít pouze pokud}
\begin{itemize}
    \item je dalším prvkem v regulačním obvodu systém typu dolní propust, který dostatečně potlačuje vyšší harmonické
    \item nelinearita při vstupním signálu $e=A\cdot sin (\omega t) $ negeneruje stejnosměrnou složku
\end{itemize}

Zjištění mezních cyklů

\includegraphics[scale = 0.5]{img/harm.rovnovha.png}

Pro určení mezních cyklů musí být systém jako je na obrázku, kde f je nelinearita a G je lin přenos typu dolní propust.

Mezní cykly v systému vzniknou pokud bude "$ F_o =-1 $", tedy pokud bude výstup systému
roven jeho výstupu posunutému o 180°, matematicky :
\begin{equation*}
    A= N_e \cdot G \cdot A
\end{equation*}
Úpravou dostaneme podmínku vzniku mezních cyklů jako :
\begin{equation*}
    G=\frac{-1}{N_e}
\end{equation*}
Tato rovnice často nelze numericky řešit a musí se řešit graficky.

\includegraphics[scale = 0.5]{img/HarmMet.png}

dál lze touto metodou určit jestli je mezní cyklus stabilní, pomocí modifikovaného Nyquistova kritéria :
například u bodu 1, pokud se sníží A dostaneme se do bodu 1'' pokud je charakteristika G nalevo do něj systém bude nestabilní tj amplituda se opět zvýší, pokud A zvýšíme dostane se do bodu 1' systém bude stabilní a amplituda se zmenší, mezní cyklus v bodě 1 je tedy stabilní.

Tupě řečeno mezní cyklus je stabilní pokud se po snížení A nachází bod 1'' napravo od G a při zvýšení A se bod 1' nachází nalevo do G.
mezní cyklus v bodě 1 je stabilní mezní cyklus v bodě 2 není.

\includegraphics[scale = 0.8]{img/stab.mez.cyklu.png}

\section{Stabilita nelineárních systémů (definice, metody vyšetření, věty o nestabilitě, stabilita uzavřené regulační
  smyčky)}

stabilitu určujeme podle Ljapunovových kritérií, ty platí pouze pro neřízené systémy.
podle Ljapunova existují následující stability :

\subsection{lokálně stabilní}
Pokud v (libovolně malém) okolí $\varepsilon $ rovnovážného stavu $x_0$ vybereme počáteční bod $\delta $ jeho trajektorie dané okolí nepustí.

\includegraphics{img/lok.stabilní.png}

podle Václavka :

\includegraphics{img/vaclave.lok.png}

\subsection{lokálně asymptoticky stabilní}
Systém je lokálně stabilní viz předchozí odstavec a zároveň platí že pro $t=\infty$ se systém dostane do $x_0$

\includegraphics[scale = 0.6]{img/lok.asypt.stab.png}

podle Václavka:

\includegraphics{img/vaclavek. lok.asypt..png}

\subsection{globálně  asymptoticky stabilní}
Pokud je systém lokálně asymptoticky stabilní a jeho okolí přitažlivosti (pokud v daném daného okolí umístíme systém skončí v $x_0$) zabírá
celý stavový prostor.

jinými slovy pro libovolný bod stavového prostoru systém skončí v $x_0$.

podle Václavka

\includegraphics{img/vaclavek.golbal.png}

Tato podmínka je docela tvrdá, proto se zavádí Praktická stabilita, systém je prakticky stabilní v oblasti $\Omega$ pokud je lokálně asymptoticky stabilní a má oblast přitažlivosti $\Omega$.

\subsection*{určení stability}
Jedná pouze o podmínky postačující pro stabilitu, pokud tedy systém podmínce nevyhoví nelze o něm říct že je nestabilní.
\subsubsection{I Ljapunovovo kritériu stability}
Určuje stabilitu systému na základě stability linearizované náhrady v okolí rovnovážného stavu.

\noindent \includegraphics[scale = 0.6]{img/Ljap.1.png}

\subsubsection{II Ljapunovovo kritériu stability}
Určíme si fukci V aby byla pozitivně
definitní (PD), spočítáme W jako gradient V (v podstatě spíš derivaci ale ve skriptech to mají nazváno gradiet). Podle toho jestli je w ND , NSD určíme stabilitu.

\includegraphics[scale = 0.6]{img/Ljap.2.png}

V můžeme volit náhodně nebo lze využít {\bf Krasovského metodou}, kde V zvolíme jako součin námi zvolené matice L a funkcí f, které získáme ze stavových rovnic.
L musí být symetrická a PD. Matematickými čáry pak dostaneme že W bude ND pokud T bude PD.
T spočítáme jako :
\begin{equation*}
    T=J^TL+JL
\end{equation*}

další metodu volby V je {\bf metoda variabilního gradientu } to ale doufám že umět nemusíme.

\subsubsection{Popovo kritérium stability}
Aby byl podle popova systém stabilní musí být splněno:
\begin{itemize}
    \item systém je neřízený
    \item systém  je v konfiguraci jak je na obr.

          \includegraphics{img/popov.schem.png}
    \item hodnota nelinerity v 0 je 0 $f(0)=0$
    \item průběh nelinearity je "pod" přímkou procházející počátkem se směrnicí k

          \includegraphics{img/popov_nelilin.png}
    \item modnifikovaná frekvenční charakteristika $f^*=Re{F}+j\omega Im{F}$ (F je frekvenční charakteristika zkoumaného systému) se musí nacházet pod popovovou přímkou,
          ta může mít libovolný sklon a musí procházet bodem -1/k

          \includegraphics{img/popov_frek.png}
\end{itemize}
pod jsou všechny podmínky splněny systém je globálně asymptoticky stabilní.

\subsection{věty o nestabilitě}
Opět se jedná pouze o postačující podmínky, pokud jim systém nevyhoví nevíme nic.

\subsubsection{I Ljapunovova věta o o nestabilitě}
pokud existuje funkce V pro kterou platí :
\begin{itemize}
    \item V(0)=0
    \item $\frac{dV}{dt}$ je PD
    \item V není ND nebo NSD v libovolně málem okolí počátku
\end{itemize}
poto je rovnovážný bod v počátku nestabilní.

\subsubsection{II Ljapunovova věta o nestabilitě}
pokud V není ND , nebo NSD, a její derivace je PSD, je rovnovážný bod v počátku nestabilní.
* V musí být spojitá a musí mít spojitou 1. derivaci.
\subsubsection{Četajevoa věta o nestabilitě}
vůbec nechápu

%------------------------
\section{Reléová regulace (on-off regulátory, řízení v klouzavém režimu).
 }

Jako regulátor je použito relé, nebo jiný spínací prvek, chovající se stejně.
Použije se pokud akční zásah nabývá jen 2 hodnot (on-off control),nebo pokud akční veličina max kladné nebo max záporné hodnoty (bang-bang control).

Nejtypičtější využití regulace teploty (žehlička, topení), tlak v kompresorech se zásobníkem.
Možné varianty relé jsou o dvě otázky výš.

Pokud bychom chtěli regulovat např teplotu pomocí klasického relé, neustále by spínalo (i vlivem šumu), proto je lepší použít relé s hysterezi, podle šířky hystereze můžeme nastavit s jakou periodou bude kmitat.

Pokud nám u řízení pomocí relé s hysterezí nevyhovuje nastavení hystereze a z nějakého (nepochopitelného) důvodu ji nemůžeme měnit lze chování upravit přidáním setrvačného článku do zpětné vazby.
Musí platit že oba přepínací body jsou větší než 0, potom při sepnutí relé dojde ke zpoždění informace ve zpětný vazbě a relé sepne zpět až později. Časová konstanta setrvačného článku musí být malá.

\includegraphics{img/rele_reg.png}

{ \it
    Pokud řídím polohu a chci stihnou zastavit včas, může být přivedená další zpětná vazba od rychlosti, touhle hodnotou si upravím kdy bude relé spínat.
}

pro návrh reléového regulátoru lze využít např Lipanova...

Pro složitější systémy se využívá {\bf řízení v klouzavém režimu}
Relé je nastaveno tak, aby se systém dostal na přepínací rozhraní  po přepínacím rozhraní došel až do 0.

\includegraphics{img/kluzavy_obr.png}

\begin{itemize}
    \item systém může obsahovat neurčitost -> jedná se o robustní řízení
    \item musím být schopen měřit aktuální hodnotu vnitřních stavů
    \item na přepínacím rozhraní kmitá relé nesmyslně rychle, proto se nahrazuje něčím pozvolnějším

          \includegraphics[scale = 0.5]{img/kluzavy_rele.png}

\end{itemize}

obeně postup návrhu řízení v klouzavém režimu
\begin{enumerate}
    \item popis systému převedeme na požadovaný tvar (oddělené neurčitosti $\delta $ od zbylých závislostí)

          \includegraphics[scale =0.4]{img/klouzavy_tavr.png}
    \item navrhneme řízení systému na který může působit přímo vstupem - tím definujeme přepínací rozhraní
    \item zavedeme proměnou co určuje odchylku od přepínacího rozhraní, její derivací získáme "systém", který musí směřovat do 0 (navrhneme pro něj takové řízení)
    \item určíme ekvivalentní řízení $u_{eq}$, tak aby v rovnici došlo k eliminaci známých členů
    \item dosadíme získáme $\Delta$
    \item určíme max $\Delta$
    \item dosadíme, určíme řízení (předpis pro regulátor) V

\end{enumerate}

%-----------------------------------
\section{Linearizace nelineárních dynamických systémů (rozvoj do Taylorovy řady, exaktní zpětnovazební
  linearizace).
 }

\subsection{Linearizace do Taylorovy řady}
\begin{itemize}
    \item nejrozšířenější
    \item linearizuje systém pouze v okolí daného bodu
    \item vstupní výstupní vektory linearizovaného systému odpovídají { \bf odchylkám} od bodu ve kterém linearizujeme
    \item pro linearizaci využijeme pouze první člen Taylorovy řady, ostatní zanedbáváme.
    \item výpočet se provádí pomocí parciálních derivací jednotlivých rovnic - Jacobiho matice
\end{itemize}

\subsection{exaktní zpětnovazební linearizace}
oproti rozvoji do Taylorovy řady popisuje celý systém, ale musíme znát hodnoty vnitřních stavů.
\subsubsection{cstup-stav}
Spočívá v tom že zavedeme nový vstup u ,ten bude osahovat opravdový vstup v + další prvky, které nahradí nelinearity.
Diferenciální r-ce je potřeba upravit tak(substituce, transformace), aby nelineární část byla jen v rovnici kde je u, potom napíšeme vhodné u.
Př.:
zadaný systém:
\begin{equation}
    \dot{x_1}=cokoliv lineárního
\end{equation}
\begin{equation*}
    \dot{x_2}=x_1+2x_2+cos(x_2)+u
\end{equation*}
řešení
\begin{equation*}
    \dot{x_1}  zůstává
\end{equation*}
\begin{equation}
    \dot{x_2}=x_1+2x_2+u
\end{equation}
\begin{equation*}
    u=v+cos(x_2)
\end{equation*}

někdy můžou být úpravy hodně složité

\subsubsection{linearizace vstup-výstup}
zajímá nás pouze jak se chová výstup v závislosti na vstupu. Postupně derivujeme výstup dokud nedostaneme závislost na vstupu.

Může dojít k tomu že aproximovaný systém bude nižšího řádu něž byl původně, tj systém bude mít skrytou vnitřní dynamiku to může působit problémy.

\includegraphics[scale =0.6]{img/vstup-výstup1}

\includegraphics[scale =0.6]{img/vstup-výstup2.png}


