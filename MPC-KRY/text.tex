\section{Základní pojmy: kryptologie, kryptografie, kryptoanalýza, princip symetrických a asymetrických šifer. Matematické základy kryptologie: modulární aritmetika, inverse "modulo", Euclidův algoritmus, Eulerova funkce, prvočísla - princip generování.Generování náhodných čísel, princip, kryptografické generátory, testování generátorů, baterie testů.}

\clearpage
\section{Symetrické šifrovací algoritmy: principy, používané techniky, Cézarova šifra, Vernamova šifra, substituční a transpoziční šifry, proudové a blokové šifry, módy blokových šifer, příklady se stručnou charakteristikou (AES, DES, RC(x)), šifry používané v GSM a UMTS, princip autentizace účastníků GSM a UMTS sítě.}

\clearpage
\section{Kvantová kryptografie: princip, protokol pro výměnu klíčů.}

\clearpage
\section{Asymetrické šifrovací algoritmy: zavazadlový algoritmus, RSA, Diffie-Hellman, El Gamal systém, šifry na bázi eliptických křivek.}

\clearpage
\section{Hašovací funkce: princip, příklady, kolize, odolnost proti kolizím.}

\clearpage
\section{Digitální podpis: DSS, DSA, PGP, struktura, používané algoritmy. PKI, certifikát X.509 struktura certifikační autorita základní části, časová razítka, autorita časových razítek}

\clearpage
\section{Autentizace, autentizační metody, vícefaktorová autentizace.}

\clearpage
\section{Možné způsoby kryptografického zabezpečení datové komunikace, možnosti zabezpečení v jednotlivých vrstvách OSI (metody a mechanizmy), bezpečný web SSL/TSL, IPsec.}

\clearpage
\section{Elektronické platby na Internetu, 3D Secure, princip, popis protokolů, použité algoritmy.}

\clearpage
\section{Postranní kanály v kryptografii, základní typy postranních kanálů.}
